\chapter{Supplementary considerations}
\label{supplementary-considerations}

%% \section{Source balance and the law of mass action}
%% \label{law-of-mass-action}

%% The conversion of precursors to tissue and the reverse process of
%% its breakdown are governed by a series of chemical reactions. The
%% stoichiometry of these reactions varies in a limited range.
%% Continuing in the simple vein adopted above, it is assumed that
%% the formation of tissue and byproducts from precursors, and the
%% breakdown of tissue, are governed by the forward and reverse
%% directions of a single reaction:
%% \begin{equation}
%% \sum\limits_{\iota=\alpha}^{\omega} n_\iota[\iota] \longrightarrow
%% [\mathrm{s}]. \label{chemreac}
%% \end{equation}

%% \noindent Here, $n_\iota$ is the (possibly fractional) number of
%% moles of species $\iota$ in the reaction. For a tissue precursor,
%% $n_\iota > 0$, and for a byproduct, $n_\iota < 0$. By the Law of
%% Mass Action for this reaction, the rate of the forward reaction
%% (number of moles of $\mathrm{s}$ produced per unit time, per unit
%% volume in $\Omega_0$) is
%% $k_\mathrm{f}\prod\limits_{\iota=\alpha}^\omega
%% [\rho_0^\iota]^{n_\iota}$, where $\prod$ on the right hand-side
%% denotes a product, not to be confused with the source, $\Pi$. The
%% rate of the reverse reaction (number of moles of $\mathrm{s}$
%% consumed per unit time, per unit volume in $\Omega_0$) is
%% $k_\mathrm{r}[\rho_0^\mathrm{s}]$, where $k_f$ and $k_r$ are the
%% corresponding reaction rates. Assuming, for the purpose of this
%% example, that the solid phase is a single compound, let the
%% molecular weight of $\mathrm{s}$ be $\sM_\mathrm{s}$. From the
%% above arguments the source term for $\mathrm{s}$ is
%% \begin{equation}
%% \Pi^\mathrm{s} =
%% \left(k_\mathrm{f}\prod\limits_{\iota=\alpha}^\omega
%% [\rho_0^\iota]^{n_\iota} -
%% k_\mathrm{r}[\rho_0^\mathrm{s}]\right)\sM_\mathrm{s},
%% \label{sourceA}
%% \end{equation}

%% \noindent Since the formation of one mole of $\mathrm{s}$ requires
%% consumption of $n_\iota$ moles of $\iota$, we have
%% \begin{equation}
%% \Pi^\iota =
%% -\left(k_\mathrm{f}\prod\limits_{\vartheta=\alpha}^\omega
%% [\rho_0^\vartheta]^{n_\vartheta} -
%% k_\mathrm{r}[\mathrm{s}]\right)n_\iota\sM_\iota, \label{sourceI}
%% \end{equation}

%% \noindent where $\sM_\iota$ is the molecular weight of species
%% $\iota$. Since, due to conservation of mass, $\sM_\mathrm{s} =
%% \sum\limits_{\iota=\alpha}^\omega n^\iota\sM_\iota$ the sources
%% satisfy $\sum\limits_{\iota=\mathrm{s},\alpha}^\omega \Pi^\iota =
%% 0$.

\section{Frame invariance and the contribution from acceleration}
\label{acceleration-objectivity}

In our earlier treatment \citep{growthpaper}, the constitutive
relation for the fluid flux had a driving force contribution arising
from the acceleration of the solid phase,
$-\rho_0^\mathrm{f}\bF^{\mathrm{T}}\frac{\partial \bV}{\partial t}$.
This term, being motivated by the reduced dissipation inequality, does
not violate the Second Law and supports an intuitive understanding
that the acceleration of the solid skeleton in one direction must result in
an inertial driving force on the fluid in the opposite
direction. However, as defined, this acceleration is obtained by the
time differentiation of kinematic quantities,\footnote{And not in terms
of acceleration {\em relative to fixed stars} for e.g., as discussed
in \cite[][Page 43]{TruesdellNoll:65}.} and does not transform in a
frame-indifferent manner. Unlike the superficially similar term
arising from the gravity vector,\footnote{Where every observer has an
implicit knowledge of the directionality of the field relative to a
fixed frame, allowing it to transform objectively. Specifically, under
a time-dependent rigid body motion imposed on the current
configuration carrying $\bx$ to $\bx^+ = \bc(t) + \bQ(t)\bx$, where
$\bc(t) \in \mathbb{R}^3$ and $\bQ(t) \in \mbox{SO}(3)$, it is
understood that the acceleration due to gravity in the transformed
frame is $\bg^+ = \bQ^\mathrm{T}\bg$ and is therefore
frame-invariant. However, $\ba^+ = \ddot{\bc} + 2\dot{\bQ}\bv +
\ddot{\bQ}\bx + \bQ\ba$ , and is therefore not frame-invariant.} the
acceleration 
term presents an improper dependence on the frame of the
observer. Thus, its use in constitutive relations is inappropriate,
and the term has been dropped in \mbox{Equation (\ref{fluidflux})}.

\todo{Cite Einstein's general relativity here.}

\section{Transport of a the fluid species: the example of an ideal
  fluid}
\label{ideal-fluid-transport}

Consider the stress divergence term
$\bF^\mathrm{T}\Bnabla\cdot\bP^\iota$. An elementary calculation
gives
\begin{equation}
\bF^\mathrm{T}\Bnabla\cdot\bP^\iota =
\Bnabla\cdot\left(\bF^\mathrm{T}\bP^\iota\right) -
\Bnabla\bF^\mathrm{T}\colon\bP^\iota. \label{stressdivI}
\end{equation}

\noindent In indicial form, where lower/upper case indices are for
components of quantities in the current/reference configuration
respectively, this relation is
\begin{displaymath}
F_{iK}P^\iota_{iJ,J} = \left(F_{iK}P^\iota_{iJ}\right)_{,J} -
F_{iK,J}P^\iota_{iJ}.
\end{displaymath}

\noindent For an ideal fluid, supporting only an isotropic Cauchy
stress, $p\bone$, we have $\bP^\mathrm{f} =
\mathrm{det}(\bF)p\bF^{-\mathrm{T}}$, where $p$ is positive in
tension. The arguments that follow assume this case. (The more
general case of a non-ideal, viscous fluid will merely have
additional terms from the viscous Cauchy stress.) The stress
divergence term is
\begin{equation}
\bF^\mathrm{T}\Bnabla\cdot\bP^\mathrm{f} =
\Bnabla\left(\mathrm{det}(\bF)p\right) -
\Bnabla\bF^\mathrm{T}\colon\bF^{-\mathrm{T}}\mathrm{det}(\bF)p,
\end{equation}

\noindent demonstrating the appearance of a hydrostatic
stress-driven contribution to ${\boldmath{\sF}}^\mathrm{f}$. This
is Darcy's Law for transport of a fluid down a pressure gradient.

For the special case of a compressible, ideal fluid we have
$e^\mathrm{f} =
\bar{e}^\mathrm{f}(\eta^\mathrm{f},\bar{\rho}^\mathrm{f})$; i.e.,
the fluid stores strain energy as a function of its \emph{current,
intrinsic} density. Fluid saturation conditions hold in biological
tissue, for which case the fluid volume fraction, $f^\mathrm{f}$,
is simply the pore volume fraction. Recall from Section
\ref{sect2} that the individual species deform with the common
deformation gradient $\bF$. Therefore the pores deform
\emph{homogeneously} with the surrounding solid phase. Physically
this corresponds to the pore size being smaller than the scale at
which the homogenization assumption of a continuum theory holds.
Momentarily ignoring changes in reference concentration of the
fluid, we have $\bF^{\mathrm{e}^\mathrm{f}} = \bF$.  Then, since
$\rho^\mathrm{f}_0 = \bar{\rho}^\mathrm{f}_0 f^\mathrm{f}$, we can
write $\hat{e}^\mathrm{f}(\bF,\eta^\mathrm{f},\rho^\mathrm{f}_0) =
\hat{e}^\mathrm{f}(\bF,\eta^\mathrm{f},\bar{\rho}^\mathrm{f}_0
f^\mathrm{f}) =
\bar{e}^\mathrm{f}(\eta^\mathrm{f},\bar{\rho}^\mathrm{f}_0/\mathrm{det}\bF)=
\bar{e}^\mathrm{f}(\eta^\mathrm{f},\bar{\rho}^\mathrm{f})$. In
this case a simple calculation shows that the hydrostatic pressure
is
\begin{displaymath}
p =
-\frac{\bar{\rho}^\mathrm{f}}{\mathrm{det}(\bF)}\frac{\partial\bar{e}^\mathrm{f}}{\partial\bar{\rho}^\mathrm{f}},
\end{displaymath}

\noindent and the stress divergence term is
\begin{displaymath}
\bF^\mathrm{T}\Bnabla\cdot\bP^\mathrm{f} =
-\Bnabla\left(\bar{\rho}^\mathrm{f}\frac{\partial\bar{e}^\mathrm{f}}{\partial\bar{\rho}^\mathrm{f}}\right)
+
\Bnabla\bF^\mathrm{T}\colon\bF^{-\mathrm{T}}\bar{\rho}^\mathrm{f}\frac{\partial\bar{e}^\mathrm{f}}{\partial\bar{\rho}^\mathrm{f}}.
\end{displaymath}

\section{Stabilisation of the simplified solute transport equation}
\label{stabilisation-solute-transport}

In weak form, the SUPG-stabilised method for
Equation~(\ref{morestdform}) is

\begin{equation}
\begin{split}
&\int_{\Omega} w^{\mathrm{h}} \left(
  \frac{\mathrm{d}\rho^{\mathrm{s}^{h}}}{\mathrm{d}t} +
  \bm^f\cdot\mathrm{grad}\left[\frac{
      \rho^{\mathrm{s}^{h}}}{\rho^f}\right] \right)
  d\Omega\\ &+\int_{\Omega} \left( \mathrm{grad}
  \left[w^{\mathrm{h}}\right] \cdot \bar{\bD^\mathrm{s}} \mathrm{grad}
  \left[ \rho^{\mathrm{s}^{h}}\right] \right)\ d\Omega\\ +&
  \sum_{\mathrm{e}=1}^{\mathrm{n_{el}}} \int_{\Omega_{\mathrm{e}}}
  \tau \frac{\bm^{f}}{\rho^f} \cdot \mathrm{grad} \left[w^{\mathrm{h}}\right] \left(
  \frac{\mathrm{d}\rho^{\mathrm{s}^{h}}}{\mathrm{d}t} +
  \bm^f\cdot\mathrm{grad}\left[\frac{
      \rho^{\mathrm{s}^{h}}}{\rho^f}\right] \right) \ d\Omega\\ -&
  \sum_{\mathrm{e}=1}^{\mathrm{n_{el}}} \int_{\Omega_{\mathrm{e}}}
  \tau \frac{\bm^{f}}{\rho^f} \cdot \mathrm{grad} \left[w^{\mathrm{h}}\right]
  \left(\mathrm{div}\left[\bar{\bD^\mathrm{s}}\ \mathrm{grad} \left[
      \rho^{\mathrm{s}^{h}}\right]\right]\right) \ d\Omega\\ = &
  \int_{\Omega} w^{\mathrm{h}} \pi^\mathrm{s} \ d\Omega +
  \int_{\Gamma_{\mathrm{h}}} w^{\mathrm{h}} h \ d\Gamma\\ +&
  \sum_{\mathrm{e}=1}^{\mathrm{n_{el}}} \int_{\Omega_{\mathrm{e}}}
  \tau \frac{\bm^{f}}{\rho^f} \cdot \mathrm{grad} \left[w^{\mathrm{h}}\right]
  \pi^\mathrm{s} \ d\Omega,
\label{stabilizedmassbal}
\end{split}
\end{equation}

\noindent where quantities with the superscript $\mathrm{h}$ represent
finite-di\-men\-sion\-al approximations of infinite-dimensional field
variables, $\Gamma_{\mathrm{h}}$ is the Neumann boundary, and this
equation introduces a numerical stabilisation parameter $\tau$, which
we have calculated from the $\mathrm{L}_{2}$ norms of element level
matrices, as described in
\cite{tezduyarsupg}.


%

% Local Variables:
% TeX-master: "thesis"
% mode: latex
% mode: flyspell
% End:
