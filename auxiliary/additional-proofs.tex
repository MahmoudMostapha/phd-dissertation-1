\chapter{Review of mathematical results}
\label{additional-proofs}

The following sections catalog some classical mathematical results
that have been frequently called-upon during the course of the
development of the theoretical formulation in
Chapters~\ref{lagrangian-perspective} and
\ref{eulerian-perspective}.\footnote{The material presented in
  Appendix~\ref{additional-proofs} is covered under the GNU Free
  Documentation License (See \href{http://www.gnu.org/copyleft/}
  {http://www.gnu.org/copyleft/}). This affords the reader the freedom
  to copy and redistribute the matter with or without modification,
  either commercially or noncommercially.}

\section{Gauss-Greens' divergence theorem}
\label{gauss-divergence}

Let $\Omega\subset \R^n$ be a bounded open set with $C^1$ boundary,
let $\Bnu_\Omega\colon \partial \Omega\to \R^n$ be the exterior unit
normal vector to $\Omega$ in the point $x$ and let $\bv\colon
\overline{\Omega}\to \R^n$ be a vector function in
$C^0(\overline\Omega,\R^n)\cap C^1(\Omega,\R^n)$. Then
\[
\int_\Omega \mathrm{div} \bv(x)\, dx =\int_{\partial \Omega} \langle
\bv(x),\Bnu_\Omega(x)\rangle \, d\sigma(x).
\]

Here, the operator $\mathrm{div} \bv$ is the
divergence of the vector field $\bv$, which is sometimes written as
$\nabla \cdot \bv$.  In the right-hand side we have a surface integral,
$d\sigma$ is the corresponding area measure on $\partial \Omega$.  The
scalar product in the second integral is sometimes written as $\bv_n(x)$
and represents the \emph{normal component} of $\bv$ with respect to
$\partial \Omega$; hence the whole integral represents the \emph{flux}
of the vector field $\bv$ through $\partial \Omega$.

The theorem as stated for the vector function $\bv$ can be extended to
the forms below:

\begin{equation*}
\int_\Omega \mathrm{grad} \bv(x)\, dx =\int_{\partial \Omega} 
\bv(x)\otimes \Bnu_\Omega(x) \, d\sigma(x),\ \phantom{\mathrm{and}}
\end{equation*}

\begin{equation*}
\int_\Omega \mathrm{curl} \bv(x)\, dx =\int_{\partial \Omega} 
\bv(x)\times \Bnu_\Omega(x) \, d\sigma(x),\ \mathrm{and}
\end{equation*}

\begin{equation*}
\int_\Omega \mathrm{div} \bT(x)\, dx =\int_{\partial \Omega} 
\bT(x)\ \Bnu_\Omega(x) \, d\sigma(x),
\end{equation*}

\noindent where  $\bT\colon\R^n \to \R^n$ is a tensor function.

\section{Reynolds' transport theorem}
\label{reynolds-transport}

\paragraph{Introduction}
{\em Reynolds' transport theorem} \citep{Reynolds:1903} is a
fundamental theorem used in formulating the basic laws of fluid
mechanics. For our purpose, let us consider a fluid flow,
characterised by its streamlines, in the Euclidean vector space
$(\mathbb{R}^3,\lVert\cdot\rVert)$ and embedded on it we consider, a
continuum body $\mathscr{B}$ occupying a volume $\mathscr{V}$ whose
particles are fixed by their material (Lagrangian) coordinates
$\mathbf{X}$, and a region $\Re$ where a control volume $\mathfrak{v}$
is defined whose points are fixed by it spatial (Eulerian) coordinates
$\mathbf{x}$ and bounded by the control surface
$\partial\mathfrak{v}$. An arbitrary tensor field of any rank is
defined over the fluid flow according to the following definition.

\begin{definition*} We call an {\em extensive tensor property} to the expression
\begin{align}
\Psi(\mathbf{x},t):=
\int_{\mathfrak{v}}\psi(\mathbf{x},t)\rho(\mathbf{x},t)dv,
\end{align}
where $\psi(\mathbf{x},t)$ is the respective {\em intensive tensor
  property}.
\end{definition*}
\paragraph{Theorem's hypothesis}
The kinematics of the continuum can be described by a diffeomorphism
$\chi$ which, at any given instant $t\in [0,\infty)\subset\mathbb{R}$,
  gives the spatial coordinates $\mathbf{x}$ of the material particle
  $\mathbf{X}$,
\begin{align*}
\mathscr{V}\times[0,\infty)\rightarrow \mathfrak{v}\times[0,\infty),
    \qquad t \mapsto t, \qquad
    \mathbf{X}\mapsto\mathbf{x}=\chi(\mathbf{X},t).
\end{align*}
Indeed the above sentence corresponds to a change of coordinates which
must verify
\begin{align*}
J=\bigg\vert\frac{\partial{x}_i}{\partial{X}_j}\bigg\vert\equiv
\big\vert{F_{ij}}\big\vert\neq{0}, \qquad
F_{ij}:=\frac{\partial{x}_i}{\partial{X}_j},
\end{align*}
$J$ being the Jacobian of transformation and $F_{ij}$ the Cartesian
components of the so-called {\em strain gradient tensor} $\mathbf{F}$.

{\bf Theorem$\quad$} The material rate of an extensive tensor property
associate to a continuum body $\mathscr{B}$ is equal to the local rate
of such property in a control volume $\mathfrak{v}$ plus the efflux of
the respective intensive property across its control surface
$\partial\mathfrak{v}$.

\begin{proof}
By taking on Eq.(1) the material time derivative,
\begin{align*}
\frac{D\Psi}{Dt}=\dot{\Psi}=\dot{\overline{\int_{\mathfrak{v}}\psi\rho\;dv}}=
\dot{\overline{\int_{\mathscr{V}}\psi\rho{J}dV}}=
\int_{\mathscr{V}}\dot{\overline{\psi\rho{J}}}dV=
\int_{\mathscr{V}}(\dot{\overline{\psi\rho}}J+\psi\rho\dot{J})dV=
\end{align*}
\begin{align*}
\int_{\mathscr{V}}\Big\{J\Big[\frac{\partial}{\partial{t}}(\psi\rho)+
  \mathbf{v}\!\cdot\!\nabla_x(\psi\rho)\Big]+
\psi\rho\;(J\nabla_x\!\cdot\!\mathbf{v})\Big\}dV=
\end{align*}
\begin{align*}
\int_{\mathscr{V}}\Big\{\Big[\frac{\partial}{\partial{t}}(\psi\rho)\Big]
+\big[\mathbf{v}\!\cdot\!\nabla_x(\psi\rho)+
  (\psi\rho)\nabla_x\!\cdot\!\mathbf{v}\big]\Big\}(JdV)
\end{align*}
\begin{align*}
=\int_{\mathfrak{v}}\frac{\partial}{\partial{t}}(\psi\rho)dv+
\int_{\mathfrak{v}}\nabla_x\!\cdot\!(\psi\rho\;\mathbf{v})dv=
\frac{\partial}{\partial{t}}\int_{\mathfrak{v}}\psi\rho\,dv+
\int_{\partial\mathfrak{v}}\psi\rho\,\mathbf{v}\!\cdot\!\mathbf{n}\,da,
\end{align*}
since $\partial_t(dv)=0$ ($\mathbf{x}$ fixed) on the first integral
and by applying the Gauss-Green's divergence theorem on the second
integral at the left-hand side. Finally, by substituting Eq.(1) on the
first integral at the right-hand side, we obtain
\begin{align}
\dot{\Psi}=\frac{\partial\Psi}{\partial{t}}+
\int_{\partial\mathfrak{v}}\psi\rho\,\mathbf{v}\!\cdot\!\mathbf{n}\,da,
\end{align}
endorsing the theorem statement.
\end{proof}

%% \todo{Prove the divergnce theorem.}

%% \todo{Make this notationally consistent with the rest of the
%%   document. Transport theorem is a generalisation of Liebniz rule.}
