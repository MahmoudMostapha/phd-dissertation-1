\chapter{Additional proofs}
\label{additional-proofs}

%% \section{Product rule of differentiation}
%% \label{product-rule}

%% We begin with two differentiable functions $f(x)$ and $g(x)$ and show
%% that their product is differentiable, and that the derivative of the
%% product has the desired form. 

%% By simply calculating, we have for all values of $x$ in the domain of
%% $f$ and $g$ that 

%% \begin{eqnarray*}
%% \frac{\D{}}{\D{x}}\left[f(x)g(x)\right]
%% & = & \lim_{h\to0}\frac{f(x+h)g(x+h) - f(x)g(x)}{h} \\
%% & = & \lim_{h\to0}\frac{f(x+h)g(x+h) + f(x+h)g(x) - f(x+h)g(x) -
%%   f(x)g(x)}{h} \\ 
%% & = & \lim_{h\to0}\left[f(x+h)\frac{g(x+h)-g(x)}{h} +
%%   g(x)\frac{f(x+h)-f(x)}{h}\right] \\ 
%% & = & \lim_{h\to0}\left[f(x+h)\frac{g(x+h)-g(x)}{h}\right] +
%% \lim_{h\to0}\left[g(x)\frac{f(x+h)-f(x)}{h}\right] \\ 
%% & = & f(x)g'(x) + f'(x)g(x).
%% \end{eqnarray*}

%% The key argument here is the next to last line, where we have used the
%% fact that both $f$ and $g$ are differentiable, hence the limit can be
%% distributed across the sum to give the desired equality.

\section{Gauss divergence theorem for higher order tensors}
\label{gauss-divergence}

\todo{Write this out for the necessary objects, and prove it.}

For a tensor valued function $\Bpsi$ which is continuously
differentiable in $\Omega$ and continuous on the closure of $\Omega$,
we have the following result:

\begin{eqnarray}
\label{eqn:euler} \int_{\Omega}  \frac{\partial \Bpsi}{\partial \bx}
\ dv &=& \int_{\partial \Omega} \Bpsi \otimes \bn\ da.
\end{eqnarray}

This is the analog of the divergence theorem for higher order
objects.

\section{Reynolds transport theorem}
\label{reynolds-transport}

\todo{Make this notationally consistent with the rest of the
  document. Transport theorem is a geeneralisation of Liebniz rule.}

\paragraph{Introduction}
{\em Reynolds transport theorem} \citep{Reynolds:1903} is a fundamental
theorem used in formulating the basic laws of fluid mechanics. For our
purpose, let us consider a fluid flow, characterized by its
streamlines, in the Euclidean vector space
$(\mathbb{R}^3,\lVert\cdot\rVert)$ and embedded on it we consider, a
continuum body $\mathscr{B}$ occupying a volume $\mathscr{V}$ whose
particles are fixed by their material (Lagrangian) coordinates
$\mathbf{X}$, and a region $\Re$ where a control volume $\mathfrak{v}$
is defined whose points are fixed by it spatial (Eulerian) coordinates
$\mathbf{x}$ and bounded by the control surface
$\partial\mathfrak{v}$. An arbitrary tensor field of any rank is
defined over the fluid flow according to the following definition.
\begin{definition*} We call an {\em extensive tensor property} to the expression
\begin{align}
\Psi(\mathbf{x},t):=
\int_{\mathfrak{v}}\psi(\mathbf{x},t)\rho(\mathbf{x},t)dv,
\end{align}
where $\psi(\mathbf{x},t)$ is the respective {\em intensive tensor property}.
\end{definition*}
\paragraph{Theorem's hypothesis}
The kinematics of the continuum can be described by a diffeomorphism $\chi$ which, at any given instant $t\in [0,\infty)\subset\mathbb{R}$, gives the spatial coordinates $\mathbf{x}$ of the material particle $\mathbf{X}$,
\begin{align*}
\mathscr{V}\times[0,\infty)\rightarrow \mathfrak{v}\times[0,\infty), \qquad
t \mapsto t, \qquad \mathbf{X}\mapsto\mathbf{x}=\chi(\mathbf{X},t).
\end{align*}
Indeed the above sentence corresponds to a change of coordinates which must verify
\begin{align*}
J=\bigg\vert\frac{\partial{x}_i}{\partial{X}_j}\bigg\vert\equiv
\big\vert{F_{ij}}\big\vert\neq{0}, \qquad
F_{ij}:=\frac{\partial{x}_i}{\partial{X}_j},
\end{align*}
$J$ being the Jacobian of transformation and $F_{ij}$ the Cartesian components of the so-called {\em strain gradient tensor} $\mathbf{F}$.
\begin{theorem*}
The material rate of an extensive tensor property associate to a continuum body $\mathscr{B}$ is equal to the local rate of such property in a control volume $\mathfrak{v}$ plus the efflux of the respective intensive property across its control surface $\partial\mathfrak{v}$.
\end{theorem*}
\begin{proof}
By taking on Eq.(1) the material time derivative,
\begin{align*}
\frac{D\Psi}{Dt}=\dot{\Psi}=\dot{\overline{\int_{\mathfrak{v}}\psi\rho\;dv}}=
\dot{\overline{\int_{\mathscr{V}}\psi\rho{J}dV}}=
\int_{\mathscr{V}}\dot{\overline{\psi\rho{J}}}dV=
\int_{\mathscr{V}}(\dot{\overline{\psi\rho}}J+\psi\rho\dot{J})dV=
\end{align*}
\begin{align*}
\int_{\mathscr{V}}\Big\{J\Big[\frac{\partial}{\partial{t}}(\psi\rho)+
\mathbf{v}\!\cdot\!\nabla_x(\psi\rho)\Big]+
\psi\rho\;(J\nabla_x\!\cdot\!\mathbf{v})\Big\}dV=
\int_{\mathscr{V}}\Big\{\Big[\frac{\partial}{\partial{t}}(\psi\rho)\Big]
+\big[\mathbf{v}\!\cdot\!\nabla_x(\psi\rho)+
(\psi\rho)\nabla_x\!\cdot\!\mathbf{v}\big]\Big\}(JdV)
\end{align*}
\begin{align*}
=\int_{\mathfrak{v}}\frac{\partial}{\partial{t}}(\psi\rho)dv+
\int_{\mathfrak{v}}\nabla_x\!\cdot\!(\psi\rho\;\mathbf{v})dv=
\frac{\partial}{\partial{t}}\int_{\mathfrak{v}}\psi\rho\,dv+
\int_{\partial\mathfrak{v}}\psi\rho\,\mathbf{v}\!\cdot\!\mathbf{n}\,da,
\end{align*}
since $\partial_t(dv)=0$ ($\mathbf{x}$ fixed) on the first integral and by applying the Gauss-Green divergence theorem on the second integral at the left-hand side. Finally, by substituting Eq.(1) on the first integral at the right-hand side, we obtain
\begin{align}
\dot{\Psi}=\frac{\partial\Psi}{\partial{t}}+
\int_{\partial\mathfrak{v}}\psi\rho\,\mathbf{v}\!\cdot\!\mathbf{n}\,da,
\end{align}
endorsing the theorem statement.
\end{proof}
%% \begin{thebibliography}{1}
%% \bibitem{cite:Reynolds}
%% O. Reynolds, {\em Papers on mechanical and physical subjects-the sub-mechanics of the Universe,} Collected Work, Volume III, Cambridge University Press, 1903.
%% \end{thebibliography}
