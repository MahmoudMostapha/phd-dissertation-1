\chapter{Source balance and the law of mass action}
\label{law-of-mass-action}

The conversion of precursors to tissue and the reverse process of
its breakdown are governed by a series of chemical reactions. The
stoichiometry of these reactions varies in a limited range.
Continuing in the simple vein adopted above, it is assumed that
the formation of tissue and byproducts from precursors, and the
breakdown of tissue, are governed by the forward and reverse
directions of a single reaction:
\begin{equation}
\sum\limits_{\iota=\alpha}^{\omega} n_\iota[\iota] \longrightarrow
[\mathrm{s}]. \label{chemreac}
\end{equation}

\noindent Here, $n_\iota$ is the (possibly fractional) number of
moles of species $\iota$ in the reaction. For a tissue precursor,
$n_\iota > 0$, and for a byproduct, $n_\iota < 0$. By the Law of
Mass Action for this reaction, the rate of the forward reaction
(number of moles of $\mathrm{s}$ produced per unit time, per unit
volume in $\Omega_0$) is
$k_\mathrm{f}\prod\limits_{\iota=\alpha}^\omega
[\rho_0^\iota]^{n_\iota}$, where $\prod$ on the right hand-side
denotes a product, not to be confused with the source, $\Pi$. The
rate of the reverse reaction (number of moles of $\mathrm{s}$
consumed per unit time, per unit volume in $\Omega_0$) is
$k_\mathrm{r}[\rho_0^\mathrm{s}]$, where $k_f$ and $k_r$ are the
corresponding reaction rates. Assuming, for the purpose of this
example, that the solid phase is a single compound, let the
molecular weight of $\mathrm{s}$ be $\sM_\mathrm{s}$. From the
above arguments the source term for $\mathrm{s}$ is
\begin{equation}
\Pi^\mathrm{s} =
\left(k_\mathrm{f}\prod\limits_{\iota=\alpha}^\omega
[\rho_0^\iota]^{n_\iota} -
k_\mathrm{r}[\rho_0^\mathrm{s}]\right)\sM_\mathrm{s},
\label{sourceA}
\end{equation}

\noindent Since the formation of one mole of $\mathrm{s}$ requires
consumption of $n_\iota$ moles of $\iota$, we have
\begin{equation}
\Pi^\iota =
-\left(k_\mathrm{f}\prod\limits_{\vartheta=\alpha}^\omega
[\rho_0^\vartheta]^{n_\vartheta} -
k_\mathrm{r}[\mathrm{s}]\right)n_\iota\sM_\iota, \label{sourceI}
\end{equation}

\noindent where $\sM_\iota$ is the molecular weight of species
$\iota$. Since, due to conservation of mass, $\sM_\mathrm{s} =
\sum\limits_{\iota=\alpha}^\omega n^\iota\sM_\iota$ the sources
satisfy $\sum\limits_{\iota=\mathrm{s},\alpha}^\omega \Pi^\iota =
0$.
