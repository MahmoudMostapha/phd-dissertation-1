\startabstractpage {A CONTINUUM THEORY OF MULTIPHASE MIXTURES FOR
  MODELLING BIOLOGICAL GROWTH} {Harish Narayanan} {Chair: Krishnakumar
  R. Garikipati}

Prompted by compelling clinical reports evidencing the pervasive role
of mechanical factors influencing growth, this dissertation presents
highlights of our modelling work aiming to gain a deeper
understanding---and the ability to better predict---the roles of these
biomechanical influences. The utility of the resulting framework
extends across disciplines by helping steer and interpret experimental
work, both under physiological and pathological cases of interest. Our
focus is currently directed toward a better understanding of the
mechanics of growing soft tissue, specifically tendon, and looking
closely at the processes that drive unbounded growth of cancerous
tissue.

{\em Growth} in biological tissue depends upon cascades of complex
biochemical reactions involving several species, as well as their
transport through the extra-cellular matrix and diffusion across cell
membranes. Our modelling effort proposes a general continuum field
formulation for growth capable of simulating this rich observed
behaviour, and proceeds to specialise it incorporating different
modelling assumptions (such as the use of an {\sl enzyme kinetics}
based growth law) to better represent cases of interest. Formulated
within the context of open system thermodynamics, the model
incorporates additional quantities pertinent to the physics of
multi-phase reacting systems, and deduces balance laws and a
constitutive framework obeying the dissipation inequality. Assumptions
central to existing mechanics theories, such as the nature of the
split of deformation maps across the different species involved, are
carefully revisited and analysed. The nonlinear partial differential
equations that arise from the theory are solved using a staggered
finite element scheme, and a reformulation of the reaction-transport
equation enabling a straightforward implementation of numerical
stabilisation at the advection-dominated limit is accentuated.

Several numerical examples are solved demonstrating the applicability
of the theory, and the computational framework presents a powerful
tool to answer specific questions---ranging from those pertinent to
viscoelastic aspects of the mechanical response of growing tendons
under different loading conditions, to quantitative investigations of
the efficacy of drugs based on how they are administered, to
understanding the cellular processes associated with tumour growth.

\thispagestyle{empty}
