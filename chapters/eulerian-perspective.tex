\chapter{An Eulerian perspective}
\label{eulerian-perspective}

 We initiate this discussion with the introduction of $\Omega_{t}$, a
 temporally-varying closure of an open set in $\mathbb{R}^{3}$ with a
 piecewise smooth boundary, which we define to be our region of
 interest. Assuming that $\Omega_{t}$ evolves in a {\em
   well-behaved}\footnote{By which we mean, there exists a
   {\color{Blue} $\mathit{C}^{2}$} (in space and time), bijective
   and orientation preserving map $\varphi_{t}(\bX):
   \mathbb{R}^3\times\mathbb{R}^{+}\cup\{0\}
   \rightarrow\mathbb{R}^{3}$ such that $\Omega_{t} = \varphi_{t}
   (\Omega_{0})$ for some {\color{Sepia} convenient}, \em{fixed}
   subset of $\mathbb{R}^{3}$, $\Omega_{0}$.} manner, our primary
 interest lies in the evolution of various field variables inside
 $\Omega_{t}$ as observed from an inertial reference frame.

\section{Preliminary definitions}
\label{eu-preliminary-definitions}

\section{Balance laws for an open mixture}
\label{eu-balance-laws}
                                   
\subsection{Balance of mass}
\label{eu-balance-of-mass}

Denoting a point in $\Omega_{t}$ by $\bx$, we turn our attention to
the evolution of the first of these field variables of interest, the
{\em concentration}\footnote{Formally defined as the mass of species
  $\iota$ per unit system volume, $\Omega_{t}$.} of an arbitrary
species $\iota$ constituting the system, $\rho^{\iota}(\bx,t):
\mathbb{R}^3\times\mathbb{R}^{+}\cup\{0\} \rightarrow
\mathbb{R}$. Assuming that $\rho^{\iota}$ is {\color{Blue}
  $\mathit{C}^{1}$} in time and {\color{Blue} $\mathit{C}^{2}$} in
space, we have from the conservation of matter for species $\iota$
over $\Omega_{t}$,

%TODO: What are the requirements for \phi^{iota}?

\begin{equation}
\underbrace{\frac{d}{dt}\left(\int_{\Omega_{t}} \rho^{\iota} dv
  \right)}_{\text{Rate of change of mass}} = 
 \underbrace{\int_{\Omega_{t}}
  \pi^{\iota} dv}_{\text{Mass being created}}
- \underbrace{\int_{\partial \Omega_{t}} \rho^{\iota}
  \left(\bv^{\iota} - \bv\right) \cdot \bn\ 
da,}_{\text{Mass leaving the domain}}
\label{globalbalanceofmass}
\end{equation}

 where $\bv(\bx,t) = \frac{\partial \varphi_{t}}{\partial t}$
is the {\em spatial velocity} of the system domain,
$\bv^{\iota}(\bx,t)$ is the spatial velocity of species
$\iota$,\footnote{Which is formally defined as $\bv^{\iota}(\bx,t) =
  \frac{\partial \varphi^{\iota}_{t}}{\partial t}$.} %TODO: Clarify
 $\pi^{\iota}(\bx,t)$ is the volumetric source of species $\iota$ and
$\bn$ is the outward normal vector over $\partial \Omega_{t}$, the
boundary of $\Omega_{t}$.

On applying Reynolds' transport theorem (Appendix~\ref{reynolds-transport}),
%TODO: Add proof in appendix

\begin{equation*}
\int_{\Omega_{t}} \frac{\partial \rho^{\iota}}{\partial t} dv
+ \cancel{\int_{\partial \Omega_{t}} \rho^{\iota} \bv \cdot \bn\ da} =
\int_{\Omega_{t}} \pi^{\iota} dv
- \int_{\partial \Omega_{t}} \rho^{\iota} \left(\bv^{\iota} -
\cancel{\bv}\right) \cdot \bn\ da,
\end{equation*}

Gauss' divergence theorem (Appendix~\ref{gauss-divergence}),
%TODO: Add proof in appendix

\begin{equation*}
\int_{\Omega_{t}} \frac{\partial \rho^{\iota}}{\partial t} dv =
\int_{\Omega_{t}} \pi^{\iota} dv
- \int_{\Omega_{t}} \mathrm{div} \left( \rho^{\iota} \bv^{\iota}\right) dv, 
\end{equation*}

and localising, we arrive at the final form of the balance of
mass of species $\iota$,

\begin{equation}
\frac{\partial \rho^{\iota}}{\partial t}  =
\pi^{\iota}
-\ \mathrm{div} \left(\rho^{\iota} \bv^{\iota}\right)
\quad \mathrm{in}\ \Omega_{t},
\label{localbalanceofmass}
\end{equation}

where $\mathrm{div} (\bullet)$ denotes the spatial divergence
operator. %TODO: Define properly, and what does this mean for the
          %field \rho^i?

%In order to establish the behaviour of the entire system, we define
%the system density $\rho$ and system velocity $\bv$ in terms of
%species quantities. 

%% =\sum_{\iota}\rho_{\iota}$ and velocity
%% of the system

%TODO: Insert the equations for the entire system here.
\subsection{The balance of momentum}
\label{eu-balance-of-momentum}

As observed from our inertial reference frame, the balance of
momentum of a species $\iota$ over $\Omega_{t}$ requires,

\begin{equation}
\begin{split}
\underbrace{\frac{d}{dt}\left(\int_{\Omega_{t}} \rho^{\iota}
  \bv^{\iota} dv \right)}_{\text{Rate of change of momentum}}  = 
& \underbrace{\int_{\Omega_{t}} \rho^{\iota} \left(\bg^{\iota} +
  \bq^{\iota}\right) dv}_{\text{Resultant body force}} 
+ \underbrace{\int_{\partial \Omega_{t}}
  \Bsigma^{\iota}\cdot\bn\ da}_{\text{Boundary traction}}\\ 
+ & \underbrace{\int_{\Omega_{t}} \pi^{\iota}\bv^{\iota}
  dv}_{\text{Momentum being created}}
- \underbrace{\int_{\partial \Omega_{t}} \left(\rho^{\iota}
  \bv^{\iota} \right) \left(\bv^{\iota} -
\bv\right) \cdot \bn\ da,}_{\text{Momentum leaving the domain}} 
\end{split}
\label{globalbalanceofmomentum}
\end{equation}

where, in addition to the quantities introduced previously,
$\bg^{\iota}(\bx,t)$ is the resultant body force of {\em external} origin
acting on species $\iota$, $\bq^{\iota}(\bx,t)$ is the resultant body
force on species $\iota$ from {\em all other species in the mixture},
and $\Bsigma^{\iota}$ is the partial Cauchy stress on species $\iota$.

Application of Reynolds' transport theorem yields
(Appendix~\ref{reynolds-transport}), 

\begin{equation*}
\begin{split}
\int_{\Omega_{t}} \frac{\partial \left(\rho^{\iota}\bv^{\iota}\right)}{\partial t} dv
+\cancel{\int_{\partial \Omega_{t}} \left(\rho^{\iota}\bv^{\iota}\right) \bv \cdot \bn\ da} =
& \int_{\Omega_{t}} \rho^{\iota} \left(\bg^{\iota}+\bq^{\iota}\right) dv 
+ \int_{\partial \Omega_{t}} \Bsigma^{\iota}\cdot\bn\ da\\
+ & \int_{\Omega_{t}} \pi^{\iota}\bv^{\iota} dv
- \int_{\partial \Omega_{t}} \left(\rho^{\iota} \left(\bv^{\iota} -
\cancel{\bv}\right) \cdot \bn \right) \bv^{\iota}\ da.
\end{split}
\end{equation*}

Using the product rule of differentiation, Gauss' divergence
theorem (Appendix~\ref{gauss-divergence}), and the balance of mass
(Equation~\ref{localbalanceofmass}), we have,

%TODO: Incorporate generalizations of the Leibniz rule
%and http://planetmath.org/encyclopedia/RelatedRates.html

%NOTE: Noting of course that this relationship needn't be satisfied
%pointwise in the code.

\begin{equation*}
\begin{split}
\cancel{\int_{\Omega_{t}} \frac{\partial \rho^{\iota}} {\partial t}
\bv^{\iota} dv} + \int_{\Omega_{t}} \rho^{\iota} \frac{\partial
  \bv^{\iota}} {\partial t} dv = 
& \int_{\Omega_{t}} \rho^{\iota} \left(\bg^{\iota}+\bq^{\iota}\right) dv 
+ \int_{\Omega_{t}} \mathrm{div}\left(\Bsigma^{\iota}\right)\ dv\\
+ & \cancel{\int_{\Omega_{t}} \pi^{\iota}\bv^{\iota} dv}
- \int_{\Omega_{t}}\left(\cancel{\mathrm{div}
\left(\rho^{\iota}\bv^{\iota}\right) \bv^{\iota}} +
\mathrm{grad}\left(\bv^{\iota}\right) \rho^{\iota}\bv^{\iota}\right)\ dv, 
\end{split}
\end{equation*}

where $\mathrm{grad} (\bullet)$ denotes the spatial gradient
operator.

Upon localisation, we obtain the final form of the balance of momentum
of species $\iota$,

\begin{equation}
\rho^{\iota} \frac{\partial \bv^{\iota}} {\partial t} = \rho^{\iota}
\left(\bg^{\iota}+\bq^{\iota}\right)  
+ \mathrm{div}\left(\Bsigma^{\iota}\right)
- \mathrm{grad}\left(\bv^{\iota}\right) \rho^{\iota}\bv^{\iota}
\quad \mathrm{in}\ \Omega_{t}.
\label{localbalanceofmomentum}
\end{equation}

%TODO: Insert the equations for the entire system here.

\subsection{The balance of angular momentum}
\label{eu-balance-of-angular-momentum}

As observed from our inertial reference frame, the balance of
angular momentum of a species $\iota$ over $\Omega_{t}$ about the
origin requires,

\begin{equation}
\begin{split}
\underbrace{\frac{d}{dt}\left(\int_{\Omega_{t}} \rho^{\iota}
  \bv^{\iota} \times \bx\ dv\right)}_{\text{Rate of change of angular  momentum}}  = 
& \underbrace{\int_{\Omega_{t}} \rho^{\iota} \left(\bg^{\iota} +
  \bq^{\iota}\right) \times \bx\ dv}_{\text{Moment from body forces}} 
+ \underbrace{\int_{\partial \Omega_{t}}
  \left(\Bsigma^{\iota}\cdot\bn\right) \times \bx\ da}_{\text{Moment from traction}}\\ 
+ & \underbrace{\int_{\Omega_{t}} \pi^{\iota}\bv^{\iota} \times
  \bx\ dv}_{\text{Angular momentum being created}} 
- \underbrace{\int_{\partial \Omega_{t}} \left(\rho^{\iota}
  \bv^{\iota}  \times \bx\right) \left(\bv^{\iota} -
\bv\right) \cdot \bn\ da,}_{\text{Angular momentum leaving the domain}} 
\end{split}
\label{globalbalanceofangularmomentum}
\end{equation}

\subsection{The balance of energy}
\label{eu-balance-of-energy}

\section{Entropy inequality and restrictions on constitutive relationships}
Use an energy that's dependent on internal variables here, allowing for visco-elasticity later.
\subsection{The Clausius-Duhem form}
\subsection{Constitutive framework}
\subsubsection{An anisotropic network model based on the elastica
  model}
\subsubsection{Forces between phases}
\subsubsection{Viscoelastic solid}
\subsubsection{Stress-dependent growth (Fg) term}
\subsubsection{Energy-dependent growth (Pi) term}
\subsubsection{Fluxes dependent on concentration gradients of other
  species}
Onsager Reciprocity
\section{Algorithmic considerations}
\subsection{The saturation constraint}
\subsection{Overall stability of the monolithic numerical scheme}
\subsection{Fluid-structure interaction boundary conditions}


%

% Local Variables:
% TeX-master: "thesis"
% mode: latex
% mode: flyspell
% End:
