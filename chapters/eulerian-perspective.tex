% 24 --- 8

%% \todo{\begin{itemize}
%%   \item Needs an explanatory figure.
%%    \item incorporating additional kinematic variables into strain energy
%%     function: derive viscoelastic response.
%%   \item Derive the growth law dependent on stress, forces between
%%     phases, source terms (even/odd), ...
%%   \item Onsager for the cancer calculation.
%%   \item Figure out how to word saturation constraint, steal and modify
%%     numerical stability analysis for monolithic scheme from proposal.
%%   \item Work from the fluid-structure interaction literature to
%%     highlight boundary conditions, or perhaps move this to the next
%%     chapter.
%% \end{itemize}}

\chapter{An Eulerian perspective}
\label{eulerian-perspective}

As detailed at the outset of Chapter~\ref{lagrangian-perspective}, the
continuum treatment presented thus far has stemmed from classical
theories for solid continua, which are traditionally formulated in a
Lagrangian setting. Since our continuum idealisation of tissues
includes fluid components, and material coordinates are, in general,
not known in fluid mechanics, this chapter revisits the derivation of
the governing field equations of growing tissues following an Eulerian
approach.

In this {\em spatial description}, attention is turned to a point in
the {\em current configuration} of the tissue, where the evolution of
field variables of interest are studied. Remarkably, this dissimilar
approach results in a set of differential equations which are
completely equivalent to those deduced in Section~\ref{balance-laws},
just {\em pushed-forward} to the current configuration. But more
significantly, the spatial approach presented below naturally leads to
a different set of primary variables more suitable to physically
relevant boundary value problems.

The outline of the chapter is as
follows. Section~\ref{eu-balance-laws} briefly recapitulates the
fundamental quantities characterising the tissue, pointing out
noteworthy differences from Section~\ref{preliminary-definitions}, and
derives the balance laws governing finite deformation
growth in terms of spatial
quantities. Section~\ref{eu-entropy-inequality} deduces constitutive
relationships consistent with the entropy inequality. Finally, certain
algorithmic considerations pertinent to the simulations presented in
Chapter~\ref{numerical-simulations-2} are discussed in
Section~\ref{eu-algorithmic-considerations}.

\section{Balance laws for an open mixture}
\label{eu-balance-laws}

 We initiate this discussion with the introduction of $\Omega_{t}$, a
 temporally-varying closure of an open set in $\mathbb{R}^{3}$ with a
 piecewise smooth boundary, which we define to be our region of
 interest. This region of interest is constructed to coincide with the
 current position of the solid component of the tissue, and our
 primary interest lies in the evolution of various field variables
 inside $\Omega_{t}$, as observed from an inertial reference frame.

It is assumed that the there exists a {$\mathit{C}^{2}$} (in space and
time), bijective and orientation preserving map $\varphi_{t}(\bX):
\mathbb{R}^3\times\mathbb{R}^{+}\cup\{0\} \rightarrow\mathbb{R}^{3}$
such that $\Omega_{t} = \varphi_{t} (\Omega_{0})$ for some convenient,
{\em fixed} subset of $\mathbb{R}^{3}$, $\Omega_{0}$. This ensures
that $\Omega_{t}$ evolves in a {\em well-behaved} manner, disallowing
non-physical deformations being imposed on the tissue, and permits the
application of Reynolds' transport theorem
(Appendix~\ref{reynolds-transport}).

Denoting a point in $\Omega_{t}$ by $\bx$, $\bv(\bx,t) =
\frac{\partial \varphi_{t}}{\partial t}$ defines the {\em spatial
  velocity} of the system domain. In contrast to
Section~\ref{balance-laws}, it is recognised at the outset that each
species of the tissue is capable of undergoing its own motion
independent of the solid component of the tissue, and so we introduce
the spatial velocity of an arbitrary species $\iota$,
$\bv^{\iota}(\bx,t)$.\footnote{Which are primary variables in
  themselves. Species velocities $\bv^{\iota}$ can be formally
  understood as $\bv^{\iota}(\bx,t) = \frac{\partial
    \varphi^{\iota}_{t}}{\partial t}$, where $\varphi^{\iota}$ is the
  deformation map of each species $\iota$ from an arbitrary reference
  configuration, but they are not explicitly tracked.} Unlike
Section~\ref{balance-of-linear-momentum}, these velocities are defined
to be the total velocities of each species, not relative velocities
with respect to the solid component.

\subsection{Balance of mass}
\label{eu-balance-of-mass}

We now turn our attention to the evolution of the first of our field
variables of interest, the species concentration of an arbitrary
species $\iota$ constituting the system, $\rho^{\iota}(\bx,t):
\mathbb{R}^3\times\mathbb{R}^{+}\cup\{0\} \rightarrow
\mathbb{R}$. These are defined as the mass of species $\iota$ per unit
{\em system volume}, $\Omega_{t}$, and thus the total {\em spatial
  density} of the tissue is given by the summation,
$\sum\limits_{\iota}\rho^\iota = \rho$. 

Assuming that $\rho^{\iota}$ is $\mathit{C}^{1}$ in time and
$\mathit{C}^{2}$ in space, we have from the conservation of matter for
species $\iota$ over $\Omega_{t}$,

\begin{equation}
\underbrace{\frac{d}{dt}\left(\int_{\Omega_{t}} \rho^{\iota} dv
  \right)}_{\text{Rate of change of mass}} = 
 \underbrace{\int_{\Omega_{t}}
  \pi^{\iota} dv}_{\text{Mass being created}}
- \underbrace{\int_{\partial \Omega_{t}} \rho^{\iota}
  \left(\bv^{\iota} - \bv\right) \cdot \bn\ 
da,}_{\text{Mass leaving the domain}}
\label{eu-globalbalanceofmass}
\end{equation}

\noindent where $\pi^{\iota}(\bx,t)$ is the volumetric source (or
sink) of species $\iota$ and $\bn$ is the outward normal vector over
$\partial \Omega_{t}$, the boundary of $\Omega_{t}$. %% The total mass
%% flux of each species $\iota$ in the current configuration is product
%% $\rho^{\iota}\bv^{\iota}$.

On applying Reynolds' transport theorem
(Appendix~\ref{reynolds-transport}) to the left-hand side,

\begin{equation*}
\int_{\Omega_{t}} \frac{\partial \rho^{\iota}}{\partial t} dv
+ \cancel{\int_{\partial \Omega_{t}} \rho^{\iota} \bv \cdot \bn\ da} =
\int_{\Omega_{t}} \pi^{\iota} dv
- \int_{\partial \Omega_{t}} \rho^{\iota} \left(\bv^{\iota} -
\cancel{\bv}\right) \cdot \bn\ da,
\end{equation*}

\noindent Gauss' divergence theorem (Appendix~\ref{gauss-divergence})
to the area integral,

\begin{equation*}
\int_{\Omega_{t}} \frac{\partial \rho^{\iota}}{\partial t} dv =
\int_{\Omega_{t}} \pi^{\iota} dv
- \int_{\Omega_{t}} \mathrm{div} \left( \rho^{\iota} \bv^{\iota}\right) dv, 
\end{equation*}

\noindent and localising, we arrive at the final form of the balance of
mass of species $\iota$,

\begin{equation}
\frac{\partial \rho^{\iota}}{\partial t}  =
\pi^{\iota}
-\ \mathrm{div} \left(\rho^{\iota} \bv^{\iota}\right)
\quad \mathrm{in}\ \Omega_{t},
\label{eu-localbalanceofmass}
\end{equation}

\noindent where $\mathrm{div} (\bullet)$ denotes the spatial divergence
operator. This result is consistent with classical mixture theory
\citep{TruesdellToupin:60} and is the analogue of
Equation~\ref{localbalanceofmass} in the current configuration.

As in Section~\ref{balance-of-mass}, when writing the mass balance
equation for a system as a whole,

\begin{equation}
\frac{\mathrm{d}}{\mathrm{d}t}\sum\limits_{\iota} \left(\int_{\Omega_{t}} \rho^{\iota} dv
  \right) = -\sum\limits_{\iota} \int_{\partial \Omega_{t}} \rho^{\iota}
  \left(\bv^{\iota} - \bv\right) \cdot \bn\ da,
\label{eu-systembalanceofmass}
\end{equation}

\noindent we treat the system as a black box, only accounting for
contributions from boundary terms and neglecting any interaction
terms. Comparing Equation~\ref{eu-systembalanceofmass} to a summation
of Equation~\ref{eu-globalbalanceofmass} over all species, it is clear
that the sources and sinks satisfy,

\begin{equation}
\sum\limits_{\iota}\pi^{\iota} = 0.
\label{eu-summationrelationmass}
\end{equation}

\subsection{The balance of momentum}
\label{eu-balance-of-momentum}

As observed from an inertial reference frame, the balance of momentum
of a species $\iota$ over $\Omega_{t}$ requires,

\begin{equation}
\begin{split}
\underbrace{\frac{d}{dt}\left(\int_{\Omega_{t}} \rho^{\iota}
  \bv^{\iota} dv \right)}_{\text{Rate of change of momentum}}  = 
& \underbrace{\int_{\Omega_{t}} \rho^{\iota} \left(\bg^{\iota} +
  \bq^{\iota}\right) dv}_{\text{Resultant body force}} 
+ \underbrace{\int_{\partial \Omega_{t}}
  \Bsigma^{\iota}\cdot\bn\ da}_{\text{Boundary traction}}\\ 
+ & \underbrace{\int_{\Omega_{t}} \pi^{\iota}\bv^{\iota}
  dv}_{\text{Momentum being created}}
- \underbrace{\int_{\partial \Omega_{t}} \left(\rho^{\iota}
  \bv^{\iota} \right) \left(\bv^{\iota} -
\bv\right) \cdot \bn\ da,}_{\text{Momentum leaving the domain}} 
\end{split}
\label{eu-globalbalanceofmomentum}
\end{equation}

\noindent where, in addition to the quantities introduced previously,
$\bg^{\iota}(\bx,t)$ is the resultant body force of {\em external} origin
acting on species $\iota$, $\bq^{\iota}(\bx,t)$ is the resultant body
force on species $\iota$ from {\em all other species in the mixture},
and $\Bsigma^{\iota}(\bx,t)$ is the partial Cauchy stress on species
$\iota$. All of these quantities are assumed to be sufficiently
smooth.

Application of Reynolds' transport theorem
(Appendix~\ref{reynolds-transport}) to
Equation~\ref{eu-globalbalanceofmomentum} yields,

\begin{equation*}
\begin{split}
\int_{\Omega_{t}} \frac{\partial \left(\rho^{\iota}\bv^{\iota}\right)}{\partial t} dv
+\cancel{\int_{\partial \Omega_{t}} \left(\rho^{\iota}\bv^{\iota}\right) \bv \cdot \bn\ da} =
& \int_{\Omega_{t}} \rho^{\iota} \left(\bg^{\iota}+\bq^{\iota}\right) dv 
+ \int_{\partial \Omega_{t}} \Bsigma^{\iota}\cdot\bn\ da\\
+ & \int_{\Omega_{t}} \pi^{\iota}\bv^{\iota} dv
- \int_{\partial \Omega_{t}} \left(\rho^{\iota} \bv^{\iota}
\right)\left(\bv^{\iota} - \cancel{\bv}\right) \cdot \bn\ da.
\end{split}
\end{equation*}

\noindent Using Leibniz rule, Gauss' divergence theorem
(Appendix~\ref{gauss-divergence}), and the balance of
mass{\footnote{Recognising that the balance of mass need not be
    satisfied exactly, pointwise in a numerical implementation.}}
(Equation~\ref{eu-localbalanceofmass}), we have,

\begin{equation*}
\begin{split}
\cancel{\int_{\Omega_{t}} \frac{\partial \rho^{\iota}} {\partial t}
\bv^{\iota} dv} + \int_{\Omega_{t}} \rho^{\iota} \frac{\partial
  \bv^{\iota}} {\partial t} dv = 
& \int_{\Omega_{t}} \rho^{\iota} \left(\bg^{\iota}+\bq^{\iota}\right) dv 
+ \int_{\Omega_{t}} \mathrm{div}\left(\Bsigma^{\iota}\right)\ dv\\
+ & \cancel{\int_{\Omega_{t}} \pi^{\iota}\bv^{\iota} dv}
- \int_{\Omega_{t}}\left(\cancel{\mathrm{div}
\left(\rho^{\iota}\bv^{\iota}\right) \bv^{\iota}} +
\mathrm{grad}\left(\bv^{\iota}\right) \rho^{\iota}\bv^{\iota}\right)\ dv, 
\end{split}
\end{equation*}

\noindent where $\mathrm{grad} (\bullet)$ denotes the spatial gradient
operator. Upon localisation, we obtain the final form of the balance
of momentum of species $\iota$, 

\begin{equation}
\rho^{\iota} \frac{\partial \bv^{\iota}} {\partial t} = \rho^{\iota}
\left(\bg^{\iota}+\bq^{\iota}\right)  
+ \mathrm{div}\left(\Bsigma^{\iota}\right)
- \mathrm{grad}\left(\bv^{\iota}\right) \rho^{\iota}\bv^{\iota}
\quad \mathrm{in}\ \Omega_{t},
\label{eu-localbalanceofmomentum}
\end{equation}

\noindent also a result consistent with classical mixture theory
\citep{TruesdellToupin:60}. This equation is the current configuration
analogue of Equation~\ref{balanceofmomentum}.

Neglecting the interaction terms, the balance of momentum for the
entire system can be written as follows,

\begin{equation}
\begin{split}
\sum\limits_{\iota} \frac{d}{dt}\left(\int_{\Omega_{t}} \rho^{\iota}
\bv^{\iota} dv 
\right) = & \sum\limits_{\iota} \left(\int_{\Omega_{t}} \rho^{\iota}
\bg^{\iota} dv + \int_{\partial \Omega_{t}} 
\Bsigma^{\iota}\cdot\bn\ da\right)\\
- &\sum\limits_{\iota}\int_{\partial \Omega_{t}} \left(\rho^{\iota}
  \bv^{\iota} \right) \left(\bv^{\iota} -
\bv\right) \cdot \bn\ da.
\end{split}
\label{eu-systembalanceofmomentum}
\end{equation}

\noindent Comparing Equation~\ref{eu-systembalanceofmomentum} to a summation
of Equation~\ref{eu-globalbalanceofmomentum} over all species, it is clear
that the sources and interaction forces satisfy the relation,

\begin{equation}
\sum\limits_{\iota}\left( \rho^{\iota}\bq^{\iota} + \pi^{\iota}
\bv^{\iota} \right) = 0.
\label{eu-summationrelationmomentum}
\end{equation}

\subsection{The balance of angular momentum}
\label{eu-balance-of-angular-momentum}

Consider the position vector $\bp(\bx)$ of a point on the tissue
relative to a fixed point\footnote{Which may or may not be
  the origin of the system's Euclidean space.} in space. The balance of angular
momentum about the fixed point, as observed from an inertial reference
frame, of a species $\iota$ over $\Omega_{t}$ requires,

\begin{equation}
\begin{split}
\underbrace{\frac{d}{dt}\left(\int_{\Omega_{t}}  \bp \times \rho^{\iota}
  \bv^{\iota}\ dv\right)}_{\text{Rate of change of angular  momentum}}  = 
& \underbrace{\int_{\Omega_{t}} \bp \times \rho^{\iota} \left(\bg^{\iota} +
  \bq^{\iota}\right)\ dv}_{\text{Moment from body forces}} 
+ \underbrace{\int_{\partial \Omega_{t}}
  \bp \times \left(\Bsigma^{\iota}\cdot\bn\right)\ da}_{\text{Moment from traction}}\\ 
+ & \underbrace{\int_{\Omega_{t}} \bp \times
  \pi^{\iota}\bv^{\iota}\ dv}_{\text{Angular momentum being created}}
\\ 
- & \underbrace{\int_{\partial \Omega_{t}} \left(\bp \times \rho^{\iota}
  \bv^{\iota}\right) \left(\bv^{\iota} -
\bv\right) \cdot \bn\ da,}_{\text{Angular momentum leaving the domain}} 
\end{split}
\label{eu-globalbalanceofangularmomentum}
\end{equation}

\noindent since it is reasonable to assume that the material
comprising the tissue is not a {\em polar material}.\footnote{Some
  materials, for e.g., liquid crystals, become polarised under the
  presence of electric fields and consequently have additional global
  torque contributions to their balance of momentum equations.}

On applying Reynolds' transport theorem
(Appendix~\ref{reynolds-transport}),
Equation~\ref{eu-globalbalanceofangularmomentum} reduces to,

\begin{equation*}
\begin{split}
\int_{\Omega_{t}} \frac{\partial}{\partial t}\left( \bp \times \rho^{\iota}
  \bv^{\iota}\right)\ dv \quad =
& \int_{\Omega_{t}} \bp \times \rho^{\iota} \left(\bg^{\iota} +
  \bq^{\iota}\right)\ dv
+ \int_{\partial \Omega_{t}}
  \bp \times \left(\Bsigma^{\iota}\cdot\bn\right)\ da\\
& + \int_{\Omega_{t}} \bp \times 
  \pi^{\iota}\bv^{\iota}\ dv - \int_{\partial \Omega_{t}} \left(\bp
  \times \rho^{\iota} 
  \bv^{\iota}\right) \bv^{\iota} \cdot \bn\ da.
\end{split}
\end{equation*}

Using Gauss' divergence theorem (Appendix~\ref{gauss-divergence}) and
Leibniz rule, we have the following relations:

\begin{displaymath}
\int_{\partial \Omega_{t}}
  \bp \times \left(\Bsigma^{\iota}\cdot\bn\right)\ da =
\int_{\Omega_{t}}
  \bp \times \mathrm{div}\left(\Bsigma^{\iota}\right)\ dv
+\int_{\Omega_{t}} \Bepsilon\colon\Bsigma^{\iota^{\mathrm{T}}} \ dv,
\end{displaymath}

\noindent where $\Bepsilon$ is the permutation symbol
(Section~\ref{balance-of-angular-momentum}),

\begin{displaymath}
\int_{\Omega_{t}}
\frac{\partial}{\partial t}\left( \bp \times \rho^{\iota}
\bv^{\iota}\right)\ dv =
\int_{\Omega_{t}} \frac{\partial \rho^{\iota}} {\partial t}
\bp \times \bv^{\iota}\ dv + \int_{\Omega_{t}} \rho^{\iota} \bp \times 
\frac{\partial \bv^{\iota}} {\partial t}\ dv,\quad\mathrm{and}
\end{displaymath}

\begin{displaymath}
\int_{\partial \Omega_{t}} \left(\bp
  \times \rho^{\iota} 
  \bv^{\iota}\right) \bv^{\iota} \cdot \bn\ da = 
\int_{\Omega_{t}}\bp \times \bv^{\iota} \mathrm{div}
\left(\rho^{\iota}\bv^{\iota}\right)\ dv + \int_{\Omega_{t}}\bp \times
\left(\mathrm{grad}\left(\bv^{\iota}\right)\rho^{\iota}\bv^{\iota}\right)\ dv,
\end{displaymath}

\noindent since $\rho^{\iota} \bv^{\iota}\times\bv^{\iota} =
0$. Substituting these relations above and invoking the balance of
mass (Equation~\ref{eu-localbalanceofmass}) and balance of linear
momentum (Equation~\ref{eu-localbalanceofmomentum}), we obtain,

\begin{equation}
\Bepsilon\colon\Bsigma^{\iota^{\mathrm{T}}} = 0,
\label{eu-localbalanceofangularmomentum}
\end{equation}

\noindent i.e., the partial Cauchy stress tensor, $\Bsigma^{\iota}$,
is symmetric. This classical result is the pushed-forward form of
the synonymous result derived earlier
(Section~\ref{balance-of-angular-momentum}) in terms of the partial
first Piola-Kirchhoff stress tensor.

When the balance of angular momentum of the entire system, deduced by
neglecting the interaction terms, is compared to the form of the
equation obtained by summation of the individual balance of angular
momenta (Equation~\ref{eu-globalbalanceofangularmomentum}) over all
species present, one obtains a relationship between the interaction
forces and source terms that is identical to
Equation~\ref{eu-summationrelationmomentum}.

\subsection{The balance of energy}
\label{eu-balance-of-energy}

\section{Entropy inequality and restrictions on constitutive
  relationships}
\label{eu-entropy-inequality}
Use an energy that's dependent on internal variables here, allowing for visco-elasticity later.
\subsection{The Clausius-Duhem form}
\subsection{Constitutive framework}
\subsubsection{Viscoelastic solid}
\subsubsection{Forces between phases}
\subsubsection{Stress-dependent growth (Fg) term}
\label{eu-stress-dependent-source}
\subsubsection{Energy-dependent growth (Pi) term}
\subsubsection{Fluxes dependent on concentration gradients of other
  species}
Onsager Reciprocity
\section{Algorithmic considerations}
\label{eu-algorithmic-considerations}
\subsection{The saturation constraint}
\subsection{Overall stability of the monolithic numerical scheme}
\subsection{Fluid-structure interaction boundary conditions}


%

% Local Variables:
% TeX-master: "thesis"
% mode: latex
% mode: flyspell
% End:
