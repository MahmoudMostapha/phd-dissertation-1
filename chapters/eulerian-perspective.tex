\chapter{An Eulerian perspective}
\label{eulerian-perspective}

As detailed at the outset of Chapter~\ref{lagrangian-perspective}, the
continuum treatment presented thus far has stemmed from classical
theories for solid continua, which are traditionally formulated in a
Lagrangian setting. Since our continuum idealisation of tissues
includes fluid components, and material coordinates are, in general,
not known in fluid mechanics, this chapter revisits the derivation of
the governing field equations of growing tissues following an Eulerian
approach.

In this {\em spatial description}, attention is turned to a region of
space coinciding with the {\em current configuration} of the tissue,
where the evolution of field variables of interest are
studied. Remarkably, this dissimilar approach results in a set of
balance equations which are completely equivalent to those deduced in
Section~\ref{balance-laws}, just {\em pushed-forward} to the current
configuration. But more significantly, the spatial approach presented
below naturally leads to the identification of a different set of
primitive variables more suitable to physically relevant boundary
value problems.

The outline of the chapter is as
follows. Section~\ref{eu-balance-laws} briefly recapitulates the
fundamental quantities characterising the tissue, pointing out
noteworthy differences from Section~\ref{balance-laws}, and derives
the balance laws governing finite deformation growth in terms of
spatial quantities. Section~\ref{eu-entropy-inequality} deduces
constitutive relationships consistent with the entropy
inequality. Finally, certain algorithmic considerations pertinent to
the simulations presented in Chapter~\ref{numerical-simulations-2} are
discussed in Section~\ref{eu-algorithmic-considerations}.

\section{Balance laws for an open mixture}
\label{eu-balance-laws}

We initiate this discussion with the introduction of $\Omega_{t}$, a
temporally-varying closure of an open set in $\mathbb{R}^{3}$ with a
piecewise smooth boundary, which we define to be our region of
interest. This region of interest is constructed to coincide with the
current position of the solid component of the tissue, and our primary
interest lies in the evolution of various field variables inside
$\Omega_{t}$, as observed from an inertial reference frame
(\citet{newton1726}, Corollary~V, p.~423).

\begin{figure}
  \centering
  \psfrag{A}{$\bX$}
  \psfrag{B}{$\Omega_0$}
  \psfrag{C}{$\Bvarphi_{t}$}
  \psfrag{D}{$\bx$}
  \psfrag{E}{$\Omega_t$}
  \psfrag{F}{$\bX^{\iota}$}
  \psfrag{G}{$\Omega_{0}^{\iota}$}
  \psfrag{H}{$\Bvarphi^{\iota}_{t}$}
  \includegraphics[width=0.8\textwidth]{images/elucidation/%
    continuum-potato-current-configuration}
  \caption{An Eulerian point of view.}
  \label{continuum-potato-current-configuration}
\end{figure}

It is assumed that the there exists a {$\mathit{C}^{2}$} (in space and
time), bijective and orientation preserving map $\Bvarphi_{t}(\bX):
\mathbb{R}^3\times\mathbb{R}^{+}\cup\{0\} \rightarrow\mathbb{R}^{3}$
such that $\Omega_{t} = \Bvarphi_{t} (\Omega_{0})$ for some
convenient, {\em fixed} subset of $\mathbb{R}^{3}$, $\Omega_{0}$. This
ensures that $\Omega_{t}$ evolves in a well-behaved manner,
disallowing non-physical deformations from being imposed on the
tissue, and furthermore, affords the application of Reynolds'
transport theorem (Appendix~\ref{reynolds-transport}). The map
$\Bvarphi_{t}$ is visualised in
Figure~\ref{continuum-potato-current-configuration}.

Denoting a point in $\Omega_{t}$ by $\bx$, $\bv(\bx,t) =
\frac{\partial \Bvarphi_{t}}{\partial t}$ defines the {\em spatial
  velocity} of the system domain. In contrast to
Section~\ref{balance-laws}, it is recognised at the outset that each
species of the tissue is capable of undergoing its own motion
independent of the solid component of the tissue; and so we introduce
the spatial velocity of an arbitrary species~$\iota$,
$\bv^{\iota}(\bx,t)$, which are assumed to be $\mathit{C}^{1}$ in
space and time.\footnote{It is important to note that these quantities
  are primitive variables in themselves. While these species
  velocities can be formally understood as \mbox{$\bv^{\iota}(\bx,t) =
    \frac{\partial \Bvarphi^{\iota}_{t}}{\partial t}$}, where
  $\Bvarphi^{\iota}_{t}$ is the deformation map of from an arbitrary
  reference configuration of each species~$\iota$ to $\Omega_{t}$
  (shown in the portion of Figure~\ref{continuum-potato-current%
    -configuration} constructed using dashed-lines), the quantities
  $\Bvarphi^{\iota}_{t}$ are neither explicitly defined nor tracked.}
Unlike the corresponding quantities defined in
Section~\ref{balance-of-linear-momentum}, these velocities are defined
to be the {\em total} velocities of each species, not velocities
relative to the solid component of the tissue.\footnote{And, with the
  redefinition of these quantities as total velocities, the velocity
  of the solid phase ceases to be special. This is manifest in the
  forms of the balance laws deduced.}

The following discussion of balance laws is carried out entirely in
terms of an arbitrary species~$\iota$, and specialisation to the solid
collagenous, fluid and solute phases (i.e. \mbox{$\iota=\mathrm{c},
  \mathrm{f}, \mathrm{s}$}) is reintroduced only when talking about
constitutive relationships specific to these different constituents of
the tissue.

\subsection{Balance of mass}
\label{eu-balance-of-mass}

We now turn our attention to the evolution of the first of our field
variables of interest, the current concentration of an arbitrary
species~$\iota$ constituting the system, $\rho^{\iota}(\bx,t):
\mathbb{R}^3\times\mathbb{R}^{+}\cup\{0\} \rightarrow
\mathbb{R}$. These are defined as the mass of species~$\iota$ per unit
       {\em system volume}, $\Omega_{t}$, and are assumed to be
       $\mathit{C}^{1}$ in time and $\mathit{C}^{2}$ in space. The
       total {\em spatial density} of the tissue can be obtained by
       their summation, $\sum\limits_{\iota}\rho^\iota = \rho$.

With these quantities defined, we have from the conservation of matter
for species $\iota$ over $\Omega_{t}$,

\begin{equation}
\underbrace{\frac{\mathrm{d}} {\mathrm{d}t}\left(\int_{\Omega_{t}}
  \rho^{\iota} \ \mathrm{d}v \right)}_{\text{Rate of change of mass}}
= \underbrace{\int_{\Omega_{t}} \pi^{\iota} \ \mathrm{d}v}_{\text{Mass
    being created}} - \underbrace{\int_{\partial \Omega_{t}}
  \rho^{\iota} \left(\bv^{\iota} - \bv\right) \cdot
  \bn\ \mathrm{d}a,}_{\text{Mass leaving the domain}}
\label{eu-globalbalanceofmass}
\end{equation}

\noindent where $\pi^{\iota}(\bx,t)$ is the volumetric source (or
sink) of species~$\iota$, which specifies the species' mass production
rate per unit system volume, and $\bn$ is the outward normal vector
over $\partial \Omega_{t}$, the boundary of $\Omega_{t}$. At this
point, the only restriction on the source terms, $\pi^{\iota}(\bx,t)$,
is that they be integrable.

On applying Reynolds' transport theorem
(Appendix~\ref{reynolds-transport}) to the left hand-side,

\begin{equation*}
\int_{\Omega_{t}} \frac{\partial \rho^{\iota}}{\partial t}
\ \mathrm{d}v + \cancel{\int_{\partial \Omega_{t}} \rho^{\iota} \bv
  \cdot \bn\ \mathrm{d}a} = \int_{\Omega_{t}} \pi^{\iota}
\ \mathrm{d}v - \int_{\partial \Omega_{t}} \rho^{\iota}
\left(\bv^{\iota} - \cancel{\bv}\right) \cdot \bn\ \mathrm{d}a,
\end{equation*}

\noindent Gauss' divergence theorem (Appendix~\ref{gauss-divergence})
to the area integral,

\begin{equation*}
\int_{\Omega_{t}} \frac{\partial \rho^{\iota}}{\partial t}
\ \mathrm{d}v = \int_{\Omega_{t}} \pi^{\iota} \ \mathrm{d}v -
\int_{\Omega_{t}} \mathrm{div} \left( \rho^{\iota} \bv^{\iota}\right)
\ \mathrm{d}v,
\end{equation*}

\noindent and localising, we arrive at the final form of the balance
of mass of species $\iota$,

\begin{equation}
\frac{\partial \rho^{\iota}}{\partial t} = \pi^{\iota} -\ \mathrm{div}
\left(\rho^{\iota} \bv^{\iota}\right) \quad \mathrm{in}\ \Omega_{t},
\label{eu-localbalanceofmass}
\end{equation}

\noindent where $\mathrm{div} (\bullet)$ denotes the spatial
divergence operator. This result is consistent with classical mixture
theory \citep{TruesdellToupin:60} and is the current configuration
analogue of Equation~(\ref{localbalanceofmass}).

As in Section~\ref{balance-of-mass}, recall that for an external
observer, the rate of change of mass of the entire system, affected
only by external agents, is independent of interconversion between
species. From the viewpoint of such an observer, the balance of mass
for the tissue as a whole reads,

\begin{equation}
\frac{\mathrm{\mathrm{d}}}{\mathrm{\mathrm{d}}t} \sum\limits_{\iota}
\left(\int_{\Omega_{t}} \rho^{\iota} \ \mathrm{d}v \right) =
-\sum\limits_{\iota} \int_{\partial \Omega_{t}} \rho^{\iota}
\left(\bv^{\iota} - \bv\right) \cdot \bn\ \mathrm{d}a.
\label{eu-systembalanceofmass}
\end{equation}

\noindent Comparing Equation~(\ref{eu-systembalanceofmass}) to a
summation of Equation~(\ref{eu-globalbalanceofmass}) over all species,
both of which being valid statements of the balance of mass of the
tissue as a whole, it is clear that the sources and sinks satisfy,

\begin{equation}
\sum\limits_{\iota}\pi^{\iota} = 0.
\label{eu-summationrelationmass}
\end{equation}

\noindent In words, the above relation states that the only kinds of
chemical interactions considered in the formulation are those in which
the masses of species get interconverted between each other, i.e.,
the system contains no additional sources of matter.

An identical relation is obtained by \citet{TruesdellToupin:60} and
others who follow their ideas (see, for e.g., \cite{passmanetal} and
\cite{ateshian07}), but their deduction of this result, and other
similar results involving quantities internal to the system
(Equations~(\ref{eu-summationrelationmomentum}) and
(\ref{eu-summationrelationenergy})), stems from a formally stated
assumption that ``the mean response of a heterogeneous mixture obeys
the ordinary equations of a continuum.'' While we do not explicitly
employ the assumption above, it is implicit in our assertion that
external observers can quantify field variables characterising species
within a system without being aware of phenomena internal to the
system.

\subsection{The balance of linear momentum}
\label{eu-balance-of-linear-momentum}

We now look at how the momentum of a species evolves under the action
of external agents, accounting for mass sources and allowing for the
possibility that matter can leave the domain.

As observed from an inertial reference frame, the balance of momentum
of a species~$\iota$ over $\Omega_{t}$ requires,

\begin{equation}
\begin{split}
\underbrace{\frac{\mathrm{d}}{\mathrm{d}t}\left(\int_{\Omega_{t}}
  \rho^{\iota} \bv^{\iota}\ \mathrm{d}v \right)}_{\text{Rate of change
    of momentum}} = & \underbrace{\int_{\Omega_{t}} \rho^{\iota}
  \left(\bg^{\iota} +
  \bq^{\iota}\right)\ \mathrm{d}v}_{\text{Resultant body force}} +
\underbrace{\int_{\partial \Omega_{t}}
  \Bsigma^{\iota}\bn\ \mathrm{d}a}_{\text{Boundary traction}}\\ +
& \underbrace{\int_{\Omega_{t}}
  \pi^{\iota}\bv^{\iota}\ \mathrm{d}v}_{\text{Momentum being created}}
- \underbrace{\int_{\partial \Omega_{t}} \left(\rho^{\iota}
  \bv^{\iota} \right) \left(\bv^{\iota} - \bv\right) \cdot
  \bn\ \mathrm{d}a,}_{\text{Momentum leaving the domain}}
\end{split}
\label{eu-globalbalanceofmomentum}
\end{equation}

\noindent where, in addition to the quantities introduced previously,
$\bg^{\iota}(\bx,t)$ is the resultant body force of {\em external}
origin acting on species $\iota$, $\bq^{\iota}(\bx,t)$ is the
resultant body force on species $\iota$ from all other species {\em in
  the mixture}, and $\Bsigma^{\iota}(\bx,t)$ is the partial Cauchy
stress on species $\iota$. The partial Cauchy stress tensor
corresponding to species~$\iota$ as the portion of the total stress
borne by the species. All of these quantities are assumed to be
sufficiently smooth.

Application of Reynolds' transport theorem (Appendix~\ref{reynolds%
  -transport}) to the left hand-side of Equation~(\ref{eu-%
  globalbalanceofmomentum}) yields,

\begin{equation*}
\begin{split}
\int_{\Omega_{t}} \frac{\partial
  \left(\rho^{\iota}\bv^{\iota}\right)}{\partial t} \ \mathrm{d}v
+\cancel{\int_{\partial \Omega_{t}}
  \left(\rho^{\iota}\bv^{\iota}\right) \bv \cdot \bn\ \mathrm{d}a} = &
\int_{\Omega_{t}} \rho^{\iota} \left(\bg^{\iota}+\bq^{\iota}\right)
\ \mathrm{d}v + \int_{\partial \Omega_{t}}
\Bsigma^{\iota}\bn\ \mathrm{d}a\\ + & \int_{\Omega_{t}}
\pi^{\iota}\bv^{\iota} \ \mathrm{d}v - \int_{\partial \Omega_{t}}
\left(\rho^{\iota} \bv^{\iota} \right)\left(\bv^{\iota} -
\cancel{\bv}\right) \cdot \bn\ \mathrm{d}a.
\end{split}
\end{equation*}

\noindent Using Leibniz's product rule, Gauss' divergence theorem
(Appendix~\ref{gauss-divergence}), and the balance of
mass{\footnote{Recognising that the balance of mass need not be
    satisfied exactly, pointwise in a numerical implementation.}}
(Equation~\ref{eu-localbalanceofmass}), we have,

\begin{equation*}
\begin{split}
\cancel{\int_{\Omega_{t}} \frac{\partial \rho^{\iota}} {\partial t}
  \bv^{\iota} \ \mathrm{d}v} + \int_{\Omega_{t}} \rho^{\iota}
\frac{\partial \bv^{\iota}} {\partial t} \ \mathrm{d}v = &
\int_{\Omega_{t}} \rho^{\iota} \left(\bg^{\iota}+\bq^{\iota}\right)
\ \mathrm{d}v + \int_{\Omega_{t}}
\mathrm{div}\left(\Bsigma^{\iota}\right)\ \mathrm{d}v\\ + &
\cancel{\int_{\Omega_{t}} \pi^{\iota}\bv^{\iota} \ \mathrm{d}v} -
\int_{\Omega_{t}}\left(\cancel{\mathrm{div}
  \left(\rho^{\iota}\bv^{\iota}\right) \bv^{\iota}} +
\mathrm{grad}\left(\bv^{\iota}\right)
\rho^{\iota}\bv^{\iota}\right)\ \mathrm{d}v,
\end{split}
\end{equation*}

\noindent where $\mathrm{grad} (\bullet)$ denotes the spatial gradient
operator. Upon localisation, we obtain the final form of the balance
of momentum of species $\iota$,

\begin{equation}
\rho^{\iota} \frac{\partial \bv^{\iota}} {\partial t} = \rho^{\iota}
\left(\bg^{\iota}+\bq^{\iota}\right) +
\mathrm{div}\left(\Bsigma^{\iota}\right) -
\mathrm{grad}\left(\bv^{\iota}\right) \rho^{\iota}\bv^{\iota} \quad
\mathrm{in}\ \Omega_{t},
\label{eu-localbalanceofmomentum}
\end{equation}

\noindent which is a result consistent with classical mixture
theory \citep{TruesdellToupin:60} and is current configuration
analogue of Equation~(\ref{localbalanceofmomentum}). 

Neglecting the interaction terms as an external observer, the balance
of momentum for the entire system can be written as follows,

\begin{equation}
\begin{split}
\sum\limits_{\iota}
\frac{\mathrm{d}}{\mathrm{d}t}\left(\int_{\Omega_{t}} \rho^{\iota}
\bv^{\iota} \ \mathrm{d}v \right) = & \sum\limits_{\iota}
\left(\int_{\Omega_{t}} \rho^{\iota} \bg^{\iota} \ \mathrm{d}v +
\int_{\partial \Omega_{t}}
\Bsigma^{\iota}\bn\ \mathrm{d}a\right)\\ -
&\sum\limits_{\iota}\int_{\partial \Omega_{t}} \left(\rho^{\iota}
\bv^{\iota} \right) \left(\bv^{\iota} - \bv\right) \cdot
\bn\ \mathrm{d}a.
\end{split}
\label{eu-systembalanceofmomentum}
\end{equation}

\noindent Comparing Equation~\ref{eu-systembalanceofmomentum} to a
summation of Equation~(\ref{eu-globalbalanceofmomentum}) over all
species, it is clear that the sources and interaction forces satisfy
the relation,

\begin{equation}
\sum\limits_{\iota}\left( \rho^{\iota}\bq^{\iota} + \pi^{\iota}
\bv^{\iota} \right) = 0,
\label{eu-summationrelationmomentum}
\end{equation}

\noindent which states that the momentum being introduced at a point
due to the creation of matter has to be negated by momentum
interactions with other species; ensuring that there is no overall
system momentum production at any point.

\subsection{The balance of angular momentum}
\label{eu-balance-of-angular-momentum}

Consider the position vector $\bp(\bx)$ of a point on the tissue
relative to a fixed point\footnote{Which may or may not be the origin
  of the system's Euclidean space.} in space. The balance of angular
momentum about $\bp$, as observed from an inertial reference frame, of
a species~$\iota$ over $\Omega_{t}$ requires,

\begin{equation}
\begin{split}
\underbrace{\frac{\mathrm{d}}{\mathrm{d}t}\left(\int_{\Omega_{t}} \bp
  \times \rho^{\iota} \bv^{\iota}\ \mathrm{d}v\right)}_{\text{Rate of
    change of angular momentum}} = & \underbrace{\int_{\Omega_{t}} \bp
  \times \rho^{\iota} \left(\bg^{\iota} +
  \bq^{\iota}\right)\ \mathrm{d}v}_{\text{Moment from body forces}} +
\underbrace{\int_{\partial \Omega_{t}} \bp \times
  \left(\Bsigma^{\iota}\bn\right)\ \mathrm{d}a}_{\text{Moment
    from traction}}\\ + & \underbrace{\int_{\Omega_{t}} \bp \times
  \pi^{\iota}\bv^{\iota}\ \mathrm{d}v}_{\text{Angular momentum being
    created}} \\ - & \underbrace{\int_{\partial \Omega_{t}} \left(\bp
  \times \rho^{\iota} \bv^{\iota}\right) \left(\bv^{\iota} -
  \bv\right) \cdot \bn\ \mathrm{d}a,}_{\text{Angular momentum leaving
    the domain}}
\end{split}
\label{eu-globalbalanceofangularmomentum}
\end{equation}

\noindent since it is reasonable to assume that the material
comprising the tissue is not a {\em polar material}.\footnote{Some
  materials, for e.g., liquid crystals, become polarised under the
  presence of electric fields and consequently have additional global
  torque contributions to their balance of momentum equations.}

On applying Reynolds' transport theorem
(Appendix~\ref{reynolds-transport}),
Equation~(\ref{eu-globalbalanceofangularmomentum}) reduces to,

\begin{equation*}
\begin{split}
\int_{\Omega_{t}} \frac{\partial}{\partial t}\left( \bp \times \rho^{\iota}
  \bv^{\iota}\right)\ \mathrm{d}v \quad =
& \int_{\Omega_{t}} \bp \times \rho^{\iota} \left(\bg^{\iota} +
  \bq^{\iota}\right)\ \mathrm{d}v
+ \int_{\partial \Omega_{t}}
  \bp \times \left(\Bsigma^{\iota}\bn\right)\ \mathrm{d}a\\
& + \int_{\Omega_{t}} \bp \times 
  \pi^{\iota}\bv^{\iota}\ \mathrm{d}v - \int_{\partial \Omega_{t}} \left(\bp
  \times \rho^{\iota} 
  \bv^{\iota}\right) \bv^{\iota} \cdot \bn\ \mathrm{d}a.
\end{split}
\end{equation*}

Using Gauss' divergence theorem (Appendix~\ref{gauss-divergence}) and
Leibniz's product rule, we have the following relations:

\begin{displaymath}
\int_{\partial \Omega_{t}} \bp \times
\left(\Bsigma^{\iota}\bn\right)\ \mathrm{d}a = \int_{\Omega_{t}}
\bp \times \mathrm{div}\left(\Bsigma^{\iota}\right)\ \mathrm{d}v
+\int_{\Omega_{t}} \Bepsilon\colon\Bsigma^{\iota^{\mathrm{T}}}
\ \mathrm{d}v,
\end{displaymath}

\noindent where $\Bepsilon$ is the permutation symbol (introduced in
Section~\ref{balance-of-angular-momentum}),

\begin{displaymath}
\int_{\Omega_{t}} \frac{\partial}{\partial t}\left( \bp \times
\rho^{\iota} \bv^{\iota}\right)\ \mathrm{d}v = \int_{\Omega_{t}}
\frac{\partial \rho^{\iota}} {\partial t} \bp \times
\bv^{\iota}\ \mathrm{d}v + \int_{\Omega_{t}} \rho^{\iota} \bp \times
\frac{\partial \bv^{\iota}} {\partial
  t}\ \mathrm{d}v,\quad\mathrm{and}
\end{displaymath}

\begin{displaymath}
\int_{\partial \Omega_{t}} \left(\bp \times \rho^{\iota}
\bv^{\iota}\right) \bv^{\iota} \cdot \bn\ \mathrm{d}a =
\int_{\Omega_{t}}\bp \times \bv^{\iota} \mathrm{div}
\left(\rho^{\iota}\bv^{\iota}\right)\ \mathrm{d}v +
\int_{\Omega_{t}}\bp \times
\left(\mathrm{grad}\left(\bv^{\iota}\right)
\rho^{\iota}\bv^{\iota}\right)\ \mathrm{d}v,
\end{displaymath}

\noindent since $\rho^{\iota} \bv^{\iota}\times\bv^{\iota} =
0$. Substituting these relations above and invoking the balance of
mass (\ref{eu-localbalanceofmass}) and balance of linear momentum
(\ref{eu-localbalanceofmomentum}), we obtain on localisation,

\begin{equation}
\Bepsilon\colon\Bsigma^{\iota^{\mathrm{T}}} = 0,
\label{eu-localbalanceofangularmomentum}
\end{equation}

\noindent i.e., the partial Cauchy stress tensor, $\Bsigma^{\iota}$,
is symmetric. This classical result is the pushed-forward form of
the synonymous result derived earlier
(Section~\ref{balance-of-angular-momentum}) in terms of the partial
first Piola-Kirchhoff stress tensor.

When the balance of angular momentum of the entire system, deduced by
neglecting the interaction terms, is compared to the form of the
equation obtained by summation of the individual balance of angular
momenta (\ref{eu-globalbalanceofangularmomentum}) over all species
present, one obtains a relationship between the interaction forces and
interconversion terms that is identical to
Equation~(\ref{eu-summationrelationmomentum}).

\subsection{The balance of energy}
\label{eu-balance-of-energy}

Recall from Section~\ref{balance-of-energy} that the internal energy
per unit mass of species~$\iota$ is denoted $e^\iota$, the heat supply
to species~$\iota$ per unit mass of that species is $r^\iota$ and the
interaction energy, $\tilde{e}^\iota$, accounts for the energy
transferred to $\iota$ by all other species, also per unit mass of
$\iota$. We now redefine the partial heat flux vector of $\iota$
defined on $\Omega_{t}$ as $\bh^\iota$, and for notational simplicity,
introduce the total energy of each species~$\iota$: $E^{\iota} =
e^\iota + \frac{1}{2} \Vert\bv^\iota\Vert^2$.

With these quantities defined, the rate of change of internal and
kinetic energies of species~$\iota$ under the action of mechanical
loads, processes of mass production and transport, heating and energy
transfer in $\Omega_{t}$ is,

\begin{equation}
\begin{split}
\underbrace{\frac{\mathrm{d}}{\mathrm{d}t}\left(\int_{\Omega_{t}}
  \rho^{\iota} E^{\iota}\ \mathrm{d}v \right)}_{\text{Rate of change
    of energy}} = & \underbrace{\int_{\Omega_{t}} \rho^{\iota}
  \left(\bg^{\iota} +
  \bq^{\iota}\right)\cdot\bv^{\iota}\ \mathrm{d}v}_{\text{Work done by
    body forces}}\quad + \underbrace{\int_{\partial \Omega_{t}}
  \left(\Bsigma^{\iota}\bn\right)\cdot\bv^{\iota}\ \mathrm{d}a}_{\text{Work
    done by boundary traction}}\\ + & \underbrace{\int_{\Omega_{t}}
  \pi^{\iota}E^{\iota}\ \mathrm{d}v}_{\text{Energy being created}} -
\underbrace{\int_{\partial \Omega_{t}} \left(\rho^{\iota} E^{\iota}
  \right) \left(\bv^{\iota} - \bv\right) \cdot
  \bn\ \mathrm{d}a}_{\text{Energy leaving the domain}} \\ +
&\underbrace{\int_{\Omega_{t}} \rho^{\iota} \left(r^{\iota} +
  \tilde{e}^{\iota} \right)\ \mathrm{d}v}_{\text{Energy supplied}} -
\underbrace{\int_{\partial \Omega_{t}}
  \bh^{\iota}\cdot\bn\ \mathrm{d}a.}_{\text{Heat flux}}
\end{split}
\label{eu-globalbalanceofenergy}
\end{equation}

On applying Reynolds' transport theorem
(Appendix~\ref{reynolds-transport}) to the left hand-side and Gauss'
divergence theorem (Appendix~\ref{gauss-divergence}) to the area
integrals, we have,

\begin{equation*}
\begin{split}
\int_{\Omega_{t}}
\frac{\partial}{\partial t}  \left(\rho^{\iota}
E^{\iota}\right)\ \mathrm{d}v  = & \int_{\Omega_{t}}
  \rho^{\iota} \left(\bg^{\iota} +
  \bq^{\iota}\right)\cdot\bv^{\iota}\ \mathrm{d}v +
\int_{\Omega_{t}}
  \left(\mathrm{div}\left(\Bsigma^{\iota}\right)\cdot\bv^{\iota} +
  \Bsigma^{\iota}\colon\mathrm{grad}\left(\bv^{\iota}\right)
  \right)\ \mathrm{d}v\\ + 
& \int_{\Omega_{t}} \pi^{\iota}E^{\iota}\ \mathrm{d}v
- \int_{\Omega_{t}} \mathrm{div}\left(E^{\iota} \rho^{\iota}
\bv^{\iota}\right)\ \mathrm{d}v
\\  + &\int_{\Omega_{t}} \rho^{\iota} \left(r^{\iota} +
  \tilde{e}^{\iota} \right)\ \mathrm{d}v
- \int_{\Omega_{t}}
  \mathrm{div}\left(\bh^{\iota}\right)\ \mathrm{d}v.
\end{split}
\end{equation*}

\noindent Upon further simplification with Leibniz's product rule, and
the balance of mass (\ref{eu-localbalanceofmass}), we are left with:

\begin{equation*}
\begin{split}
\int_{\Omega_{t}}
\rho^{\iota} \frac{\partial E^{\iota}}{\partial t}\ \mathrm{d}v  = & \int_{\Omega_{t}}
  \rho^{\iota} \left(\bg^{\iota} +
  \bq^{\iota}\right)\cdot\bv^{\iota}\ \mathrm{d}v +
\int_{\Omega_{t}}
  \left(\mathrm{div}\left(\Bsigma^{\iota}\right)\cdot\bv^{\iota} +
  \Bsigma^{\iota}\colon\mathrm{grad}\left(\bv^{\iota}\right)
  \right)\ \mathrm{d}v\\ 
& - \int_{\Omega_{t}} \rho^{\iota}\bv^{\iota}
\cdot \mathrm{grad}\left( E^{\iota} \right)\ \mathrm{d}v
\\  + &\int_{\Omega_{t}} \rho^{\iota} \left(r^{\iota} +
  \tilde{e}^{\iota} \right)\ \mathrm{d}v
- \int_{\Omega_{t}}
  \mathrm{div}\left(\bh^{\iota}\right)\ \mathrm{d}v.
\end{split}
\end{equation*}

We now expand the total energy of each species~$\iota$, $E^{\iota} =
e^\iota + \frac{1}{2} \Vert\bv^\iota\Vert^2$, and take the
derivatives,

\begin{equation*}
\begin{split}
\int_{\Omega_{t}}
\rho^{\iota} \left(\frac{\partial e^{\iota}}{\partial
  t} + \bv^{\iota}\cdot\frac{\partial \bv^{\iota}}{\partial
  t}\right)\ \mathrm{d}v  = & \int_{\Omega_{t}} 
  \rho^{\iota} \left(\bg^{\iota} +
  \bq^{\iota}\right)\cdot\bv^{\iota}\ \mathrm{d}v\\& +
\int_{\Omega_{t}}
  \left(\mathrm{div}\left(\Bsigma^{\iota}\right)\cdot\bv^{\iota} +
  \Bsigma^{\iota}\colon\mathrm{grad}\left(\bv^{\iota}\right)
  \right)\ \mathrm{d}v\\ 
& - \int_{\Omega_{t}} \rho^{\iota}\bv^{\iota} \cdot
  \left(\mathrm{grad}\left(e^{\iota}\right)
  +\mathrm{grad}(\bv^{\iota})^{\mathrm{T}} \bv^{\iota}\right)\ \mathrm{d}v
\\  + &\int_{\Omega_{t}} \rho^{\iota} \left(r^{\iota} +
  \tilde{e}^{\iota} \right)\ \mathrm{d}v
- \int_{\Omega_{t}}
  \mathrm{div}\left(\bh^{\iota}\right)\ \mathrm{d}v.
\end{split}
\end{equation*}

\noindent Finally, we apply the balance of linear
momentum~(\ref{eu-localbalanceofmomentum}) and localise to arrive at,

\begin{equation}
\rho^{\iota} \frac{\partial e^{\iota}}{\partial t} =
\Bsigma^{\iota}\colon\mathrm{grad}\left(\bv^{\iota}\right) +
\rho^{\iota} \left(r^{\iota} + \tilde{e}^{\iota} \right) -
\mathrm{div}\left(\bh^{\iota}\right) - \rho^{\iota}\bv^{\iota} \cdot
\mathrm{grad} \left(e^{\iota}\right) \quad
\mathrm{in}\ \Omega_{t},
\label{eu-localbalanceofenergy}
\end{equation}

\noindent the form of the balance of energy of a species~$\iota$ in
$\Omega_{t}$ which is most convenient for combining with the entropy
inequality, leading to the Clausius-Duhem form of the dissipation
inequality in the following section.

In order to obtain a relationship between the momentum transfer, mass
interconversion and inter-species energy transfer terms within the
system, we first express the rate of change of energy of the system
interacting with its environment from the point of view of an external
observer unaware of these internal interactions as:

\begin{equation}
\begin{split}
\sum\limits_{\iota}
\frac{\mathrm{d}}{\mathrm{d}t}\left(\int_{\Omega_{t}} \rho^{\iota}
E^{\iota}\ \mathrm{d}v \right) = & \sum\limits_{\iota}
\int_{\Omega_{t}} \rho^{\iota} \bg^i\cdot\bv^{\iota}\ \mathrm{d}v +
\sum\limits_{\iota} \int_{\partial \Omega_{t}}
\left(\Bsigma^{\iota}\bn\right)\cdot\bv^{\iota}\ \mathrm{d}a\\ -&
\sum\limits_{\iota} \int_{\partial \Omega_{t}} \left(\rho^{\iota}
E^{\iota} \right) \left(\bv^{\iota} - \bv\right) \cdot
\bn\ \mathrm{d}a \\ + & \sum\limits_{\iota} \int_{\Omega_{t}}
\rho^{\iota} r^{\iota}\ \mathrm{d}v - \sum\limits_{\iota}
\int_{\partial \Omega_{t}} \bh^{\iota}\cdot\bn\ \mathrm{d}a.
\end{split}
\label{eu-energyexternal}
\end{equation}
 
\noindent Since the equation above and a summation of Equation~%
\ref{eu-globalbalanceofenergy} over all $\iota$ are equivalent
statements of the balance of energy of the entire system, it follows
upon inspection and localisation that,

\begin{equation}
 \sum\limits_{\iota} \left( \rho^{\iota}\bq^{\iota}\cdot\bv^{\iota} +
 \pi^{\iota} E^{\iota} + \rho^{\iota}\tilde{e}^{\iota} \right) = 0.
\label{eu-summationrelationenergy}
\end{equation}


\section{Entropy inequality and restrictions on constitutive
  relationships}
\label{eu-entropy-inequality}

In contrast with Section~\ref{entropy-inequality}, we assume that the
observer is external to the system and knows nothing about internal
interactions. 

Use an energy that's dependent on internal variables here, allowing for visco-elasticity later.
\subsection{The Clausius-Duhem form}
\subsection{Constitutive framework}
\label{eu-constitutive-framework}
\subsubsection{Viscoelastic solid}
\label{eu-viscoelastic-solid}

%% \noindent{\bf Remark 3}: Since soft biological tissues usually
%% demonstrate rate-dependent response, it has been common to employ
%% a solid viscoelastic constitutive model for them. This approach
%% fits within our framework, with a modification of the internal
%% energy to include its dependence upon internal variables that
%% represent the viscoelastic stress-like parameters. However, a more
%% physiologically-valid model may be one with a purely hyperelastic
%% solid phase, and a viscous fluid. In such a composite model the
%% rate-dependent behavior would arise from the fluid.
\subsubsection{Forces between phases}
\subsubsection{Stress-dependent growth (Fg) term}
\label{eu-stress-dependent-source}
Tie this to Wolff's law if possible.
\subsubsection{Energy-dependent growth (Pi) term}
\subsubsection{Fluxes dependent on concentration gradients of other
  species}
\label{onsager-reciprocity}
Onsager Reciprocity
\section{Algorithmic considerations}
\label{eu-algorithmic-considerations}
\subsection{The saturation constraint}
% Work on this from ``A Theory of Multiphase Mixtures''
%in constrast to this, where we impose it all the time
\subsection{Overall stability of the monolithic numerical scheme}
\subsection{Fluid-structure interaction boundary conditions}

%% Kinematics of growth (absence) does not need the use of \bF and
%% this differentiates this chapter from 2.

%% \todo{\begin{itemize}
%%   \item Needs an explanatory figure.
%%    \item incorporating additional kinematic variables into strain energy
%%     function: derive viscoelastic response.
%%   \item Derive the growth law dependent on stress, forces between
%%     phases, source terms (even/odd), ...
%%   \item Onsager for the cancer calculation.
%%   \item Figure out how to word saturation constraint, steal and modify
%%     numerical stability analysis for monolithic scheme from proposal.
%%   \item Work from the fluid-structure interaction literature to
%%     highlight boundary conditions, or perhaps move this to the next
%%     chapter.
%% \end{itemize}}

% The move to the following perspective was motivated by the fact that
% the earlier formulation, and corresponding numerical examples (cite
% sections on role of mass balance in current conf., the deformation
% gradient split for the fluid, and the numerical example involving
% assumptions on deformation gradient of the fluid) involved
% quantities that, while reasonable from a 
% mathematical pint of view, did not correspond to physical quantities
% of interest in terms of physiological bvps. Point out that it's been
% 2--3 years of refining our understanding and trials with the
% previous code that's led to this point?

%

% Local Variables:
% TeX-master: "thesis"
% mode: latex
% mode: flyspell
% End:
