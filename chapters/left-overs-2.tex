%%   As
%% discussed in Section \ref{sect1}, the solid tissue phase has only
%% a mass source/sink associated with it, and no flux. The fluid
%% tissue phase has only a mass flux, and no
%% source/sink\footnote{Considering the case of the lymphatic fluid,
%% this implies that lymph glands are assumed not to be present.}.

%% Since the solid phase, $\mathrm{s}$, does not undergo transport,
%% its motion is specified entirely by $\Bvarphi(\bX,t)$. We describe
%% the remaining species $\mathrm{f},\alpha,\dots,\omega$ as
%% convecting with the solid phase and diffusing with respect to it.
%% They therefore have a velocity relative to $\mathrm{s}$. Since the
%% remaining species convect with $\mathrm{s}$, it implies a local
%% homogenization of deformation. The modelling assumption is made
%% that at each point, $\bX$, the individual phases undergo the same
%% deformation.

%% The concentrations,
%% $\rho_0^\iota$, change as a result of mass transport and
%% inter-conversion of species, implying that the total density in
%% the reference configuration, $\rho_0$, changes with time. They are
%% parameterized as $\rho_0^\iota(\bX,t)$.

%% We define concentrations of the species
%% $\rho_0^\iota=\bar{\rho}_0^\iota f^\iota$ as masses per unit
%% volume in $\Omega_0$. The intrinsic species density is
%% $\bar{\rho}_0^\iota$, and $f^\iota$ is the volume fraction of
%% $\iota$, for $\iota = \mathrm{s,f},\alpha,\dots,\omega$. In an
%% experiment it is far easier to measure the concentration,
%% $\rho_0^\iota$, rather than the intrinsic species density,
%% $\bar{\rho}_0^\iota$\footnote{In our experiments we have measured
%% the mass concentration $\rho_0^\iota$ of collagen in engineered
%% tendons grown \emph{in vitro}. These results will be presented
%% elsewhere \citep{Calve:04}.}. The concentrations also have
%% the property $\sum\limits_{\iota}\rho_0^\iota = \rho_0$, the total
%% material density of the tissue, with the sum being over all
%% species $\mathrm{s,f},\alpha,\dots,\omega$.

%% \noindent Since the solid phase of the tissue does not undergo
%% mass transport, there is no associated flux. Localizing the result
%% gives

%% \begin{equation}
%% \frac{\partial\rho_0^\mathrm{s}}{\partial t} = \Pi^\mathrm{s},
%% \label{massballocA}
%% \end{equation}
%% \noindent where the explicit dependence upon position and time has
%% been suppressed.

%% The fluid phase of the tissue, $\mathrm{f}$, may be thought of as
%% the interstitial or lymphatic fluid that perfuses the tissue. As
%% explained above, we do not consider sources of fluid in the region
%% of interest. The fluid therefore enters and leaves $\Omega_0$ as a
%% flux, $\bM^\mathrm{f}$. The balance of mass in integral form is
%% \begin{equation}
%% \frac{\mathrm{d}}{\mathrm{d}t} \int\limits_{\Omega_0}
%% \rho_0^\mathrm{f} (\bX,t)\mathrm{d}V =
%% -\int_{\partial\Omega_0}\bM^\mathrm{f}(\bX,t)\cdot\bN \mathrm{d}A,
%% \label{massbalintB}
%% \end{equation}

%% Applying the Divergence Theorem to the surface
%% integral and localizing the result gives
%% \begin{equation}
%% \frac{\partial\rho_0^\mathrm{f}}{\partial t} = -
%% \Bnabla\cdot\bM^\mathrm{f}, \label{massballocB}
%% \end{equation}

%% \noindent where $\Bnabla(\bullet)$ is the gradient operator
%% defined on $\Omega_0$, and $\Bnabla\cdot(\bullet)$ denotes the
%% divergence of a vector or tensor argument on $\Omega_0$.

%% For the precursor and byproduct species,
%% $\iota=\alpha,\dots,\omega$, the balance of mass in integral form
%% is
%% \begin{equation}
%% \frac{\mathrm{d}}{\mathrm{d}t} \int\limits_{\Omega_0} \rho_0^\iota
%% (\bX,t)\mathrm{d}V = \int\limits_{\Omega_0} \Pi^\iota
%% (\bX,t)\mathrm{d}V
%% -\int_{\partial\Omega_0}\bM^\iota(\bX,t)\cdot\bN \mathrm{d}A.
%% \label{massbalintI}
%% \end{equation}

%% \noindent In local form it is
%% \begin{equation}
%% \frac{\partial\rho_0^\iota}{\partial t} = \Pi^\iota -
%% \Bnabla\cdot\bM^\iota,\;\forall\,\iota=\alpha,\dots,\omega.
%% \label{massballocI}
%% \end{equation}

%% \noindent Of course, this last equation is the general form of
%% mass balance for any species $\iota$, recalling that in
%% particular, $\bM^\mathrm{s} = \bzero$ and $\Pi^\mathrm{f} = 0$.
%% This form will be used in the development that follows.

%% The fluxes,
%% $\bM^\iota,\;\forall\,\iota=\mathrm{f},\alpha,\dots,\omega$,
%% represent mass transport of the fluid, of precursors to the
%% reaction site, and of byproducts from sites of tissue breakdown.
%% The sources,
%% $\Pi^\iota,\;\forall\iota=\mathrm{s},\alpha,\dots,\omega$, arise
%% from inter-conversion of species. The sources/sinks in
%% (\ref{massballocA}) and (\ref{massballocI}) are therefore related,
%% as tissue and byproducts are formed by consuming precursors (amino
%% acids and nutrients, for instance). To maintain a degree of
%% simplicity in this initial exposition, we will restrict our
%% description of tissue breakdown to the reverse of this reaction.

%% %% The behaviour of the entire system can be determined by summing
%% %% \mbox{Equation~(\ref{massbalance1})} over all species $\iota$.
%% %% Additionally, sources and sinks satisfy the relation

%% %% \begin{equation}
%% %% \sum\limits_\iota\Pi^\iota = 0, \label{sourcebalance}
%% %% \end{equation}

%% %% \noindent which is consistent \citep{growthpaper} with the Law of Mass
%% %% Action for reaction rates and with Mixture Theory
%% %% \citep{TruesdellNoll:65}.


%% The body undergoes deformation, $\Bvarphi(\bX,t)$, and has a
%% material velocity field $\bV(\bX,t) =
%% \partial\Bvarphi(\bX,t)/\partial t$. In discussing momentum and energy, it proves convenient to define a
%% material velocity of species $\iota$ relative to the solid phase
%% as $\bV^\iota = (1/\rho_0^\iota)\bF\bM^\iota$. Recall (from
%% Section \ref{sect2}) that the remaining species are described as
%% deforming with the solid phase and diffusing relative to it.
%% Therefore $\bF$ is common to all species. The spatial velocity
%% corresponding to $\bV^\iota$ is $\bv^\iota =
%% (1/\rho^\iota)\bm^\iota = \bV^\iota$, by the Piola transform.
%% Since fluxes are defined relative to the solid tissue phase, which
%% does not diffuse, the total material velocity of the solid phase
%% is $\bV$, and for each of the remaining species it is $\bV +
%% \bV^\iota$, $\iota = \mathrm{f},\alpha,\dots,\omega$. Formally, we
%% can write the material velocity as  $\bV + \bV^\iota$, $\iota =
%% \mathrm{s},\mathrm{f},\alpha,\dots,\omega$ with the understanding
%% that $\bV^\mathrm{s} = \bzero$. Likewise, $\Pi^\mathrm{f} = 0$.
%% This convention has been adopted in the remainder of the paper.

%% We first write the balance of linear momentum in
%% the reference configuration, $\Omega_0$. 


%% %% The total first Piola-Kirchhoff stress tensor, $\bP$, is the sum of
%% %% the partial stresses $\bP^{\,\iota}$ (borne by a species $\iota$) over all
%% %% the species present.

%% %% With the introduction 
%% %%   of these quantities, the 
%% %% balance of linear momentum in local form over $\Omega_0$ for solid
%% %% collagen and fluid is,

%% The mass fluxes, $\bM^\iota$, and mass sources, $\Pi^\iota$ make
%% important contributions to the balance of linear momentum, as shown
%% below.
