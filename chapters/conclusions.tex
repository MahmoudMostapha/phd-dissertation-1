\chapter{Concluding remarks}
\label{conclusions}

This dissertation presented a number of significant enhancements to
our original growth formulation presented in \citet{growthpaper}.
With the reformulation of the theory in an Eulerian setting in
Chapter~\ref{eulerian-perspective}, the governing equations for the
fluid phase are now recast in terms of the fluid velocity and
pressure; primitive variables that are natural to fluid
mechanics. This, along with the revised computational formulation that
solves the momentum balance equations for each species separately,
retaining all aspects of the coupling between the equations, has
rapidly led to the establishment of a practical environment to study
the biphasic mechanics response of soft tissues. While the
computational examples presented in this dissertation focused mostly
on the interaction between elastic phases, it has proven
straightforward to extend it to more involved cases, such as studying
the flow of a viscous fluid through a porous, hyperelastic
solid. Thus, it potentially provides a means to systematically explore
the various sources for rate-dependent behaviour of soft tissues: the
viscoelastic deformation of the collagen network, the viscosity of the
extra-cellular fluid, frictional effects arising from stress-driven
fluid flow through the network, as well as suitable combinations of
these effects.

Carefully revisiting the assumptions underlying the behaviour of
mixtures, and taking a close look anew at the Dissipation-Inequality
in an attempt at constitutive specification to satisfy it a priori
brought to light some new, key, constitutive relations. This is
another significant aspect of this work.

One key relationship was for the mass interconversion terms that
suggested that it is the energetic difference between the reactants
and products of a chemical reaction that drive the reaction forward; a
well-established concept in chemistry. This broad guideline allows for
the specification of a very general class of source terms representing
varying kinds of biochemistry, yet retaining consistency with the
thermodynamics.

Perhaps the most exciting result that arose from this analysis of the
implications of the Dissipation-Inequality was the dissipation-driven
growth tensor presented in
Section~\ref{eu-stress-dependent-growth}. Experimental work on wound
healing in ligaments \citep{Provenzanoetal:2003} and stress-based
inhibition of tumours \citep{jain1997} seems to suggest that the
processes of growth are dependent upon the local stress field in a
related manner. However, a direct correlation with experiment, if
found, will be a manifestation in Biology a phenomena that is common
in Materials Physics.

%

% Local Variables:
% TeX-master: "thesis"
% mode: latex
% mode: flyspell
% End:
