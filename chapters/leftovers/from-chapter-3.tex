%% %% The fluid mass balance equation is solved in the current
%% %% configuration, i.e. (\ref{massbalcurr}) for $\iota = \mathrm{f}$, but 
%% %% mass balance for the solid collagenous phase is solved in the
%% %% reference configuration, i.e. (\ref{massbalance1}) for $\iota =
%% %% \mathrm{c}$. 

%% %% The balance of linear momentum
%% %% that we solve is (\ref{linearmombalance}) summed for $\iota =
%% %% \mathrm{c,f}$, with the constraint in (\ref{qrelation}) imposed.



%% \begin{figure}[ht]
%%   \centering
%%   \psfrag{BC}{\small Concentration boundary condition}
%%   \psfrag{RC}{\small Reduced concentration}
%%   \includegraphics[width=0.4\textwidth]
%%                   {images/elucidation/reference-boundary-conditions}
%%   \caption{Constant concentration boundary conditions in the reference
%%     configuration.} 
%%   \label{erroneous-bc}
%% \end{figure}

%% Relations (\ref{\ref{masssummationresult}}),
%% (\ref{momentumsummationresult}) and (\ref{energysummationresult})
%% relating the interaction terms are used to reduce the system of
%% equations when solving combined tissue-level problems

%%  In the numerical our numerical simulations that appear below in
%% Section~\ref{}, the numerical values used for
%% the parameters introduced in (\ref{wlcmeq}) are based on those in
%% \citet{kuhlremod05}.

%% Point to appendix, if it will exist.


%% \noindent{\bf Remark 4}: The constitutive relations
%% (\ref{stress-constrelI}) and (\ref{MIconstrel}) respectively
%% specify the partial stress, $\bP^\iota$, and flux, $\bM^\iota$, of
%% a species. The flux also implies the relative velocity,
%% $\bV^\iota$. The velocity of the solid phase, $\bV$ is obtained
%% from the local form of the balance of linear momentum for the
%% system (\ref{linmombalsys}). With all these quantities known, the
%% individual interaction forces between species,
%% $\rho^\iota_0\bq^\iota$, can be obtained from
%% (\ref{ballinmomrefI}). They are, however, not needed while solving
%% for the balance of linear momentum of the system.

%% The theory developed in Sections \ref{sect2}--\ref{sect5} has been
%% implemented in a computational formulation, retaining much of the
%% complexity of the coupled balance laws and constitutive relations.
%% For realistic soft tissue material parameters, the contribution of
%% the fluxes and interaction forces between species to the balance
%% of linear momentum of the composite tissue is negligible. This
%% simplification has been used. As a preliminary demonstration of
%% the theory\footnote{This numerical section has been included
%% mainly for completeness of this theoretical paper. A separate
%% paper, currently in preparation, will present the computational
%% formulation and contain a detailed examination of a number of
%% initial and boundary value problems for growth.}, we present a
%% computation of the coupled physics in the early stages of uniaxial
%% extension of a cylindrical soft tissue specimen. The motivation
%% for this model problem comes from our experimental model of
%% engineered, functional tendon constructs grown \emph{in vitro},
%% having the same cylindrical geometry. The experimental aspects of
%% our broad-based project on soft tissue growth are described
%% elsewhere \citep{Calve:04}. In addition to engineering
%% scaffold-less tendon constructs from neonatal rat fibroblast
%% cells, we have the ability to impose a range of mechanical,
%% chemical, nutritional and electrical stimuli on them and study the
%% tissue's response. Besides modelling these experiments, the
%% mathematical formulation described here presents researchers with
%% a vehicle for testing scenarios and framing hypotheses that can be
%% experimentally-validated in our laboratory, thereby driving the
%% experimental studies.

%% %\subsection{Boundary and initial conditions; coupled solution method}
%% %\label{}

%% Boundary conditions for mass transport consisted of the specified
%% fluid concentration at all external surfaces of the cylinder. This
%% value was fixed at $500\,\mathrm{kg.m}^{-3}$. With these boundary
%% conditions the fluid flux normal to surfaces of the specimen is
%% determined by solving the initial and boundary value problem. The
%% bottom planar surface was fixed in the $\be_3$ direction and a
%% displacement was applied at the top surface, also in the $\be_3$
%% direction, to give a nominal strain rate of
%% $0.05\,\mathrm{sec}^{-1}$ in the $\be_3$ direction. This is the
%% only mechanical load on the problem. Initial conditions were
%% $\rho_0^\mathrm{f}(\bX,0) =
%% 500\,\mathrm{kg.m}^{-3},\;\rho_0^\mathrm{s}(\bX,0) =
%% 500\,\mathrm{kg.m}^{-3}$, and for the mechanical problem,
%% $\bu(\bX,0) = \bzero,\,\bV(\bX,0) = \bzero$.

%% The coupled problem was solved by a staggered scheme based upon
%% operator splits
%% . The details 
%% will be presented in a future communication that will focus upon
%% computational aspects and numerical examples. Here we only mention
%% that the staggered scheme consists of identifying the displacement
%% and species concentrations as primitive variables associated with
%% the mechanical and mass transport problems. The mechanical problem
%% is solved holding the concentrations fixed. The resulting
%% displacement field is then held constant to solve the mass
%% transport problem. The transient solution is obtained for
%% mechanics using energy-momentum conserving schemes
%% \citep{SimoTarnow:1992b,SimoTarnow:1992a,Gonzalezphd:1996}, and
%% for mass transport using the Backward Euler Method. Hexahedral
%% elements are employed, combined with nonlinear projection methods
%% \citep{simotaylorpister:85} to treat the near-incompressibility
%% imposed by the fluid. The numerical formulation has been
%% implemented within the nonlinear finite element program, FEAP
%% \citep{feapmanual}.

%% --------------



%% Its specialization to the case discussed in this section is straighforward. The particular
%% choice of primitive unknowns used in this section can be shown to ensure unconditional stability
%% of this operator split for parabolic 
%% uid transport equations that are equivalent to our formulation,
%% 2Fickean diusion manifests itself as a concentration gradient-driven 
%% ux. Even in the absence of Fickean diusion, a
%% concentration gradient 
%% ux remains, since the 
%% uid stress is concentration-dependent.
%% 6
%% and coupled with nite strain elasticity (Armero, 1999). This was borne out in our computations, with
%% somewhat slow convergence with respect to number of passes, due to the strong coupling of this problem.
%% (One pass is completed when all equations have been solved once.) In Section 5 we demonstrate how
%% this analysis can be extended to the reformulation of the growth problem, by solution of the detailed
%% solid and 
%% uid momentum balance equations, that is being advocated in this proposal.

%% Identifying the displacement fields, $\bu$, and
%% the concentration fields, $\rho^{\iota}$, as primitive variables
%% associated with the mechanical and mass transport problems, each
%% equation is solved for their respective variable holding all other
%% variables fixed, in turn. 

%%  the
%% balance of momentum is solved holding the concentrations fixed, and
%% the resulting displacement field is then held constant to solve the
%% balance of mass equations. Convergence of the system of equations is
%% attained when the solution of each set of equations has converged to
%% within a 
