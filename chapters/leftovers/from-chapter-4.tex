%% The total mass 
%% flux of each species $\iota$ in the current configuration is product
%% $\rho^{\iota}\bv^{\iota}$.

%% Recall that in Section~\ref{balance-of-angular-momentum}, when
%% observing the tissue from a Lagrangian perspective, the partial Cauchy
%% stresses, $\Bsigma^{\iota}$, were found to be symmetric. This section
%% derives the result in terms of spatial quantities.

%% \noindent Motivated by the last term in (\ref{reduced-dissipation%
%%   -fluid}), we specify a frictional force on the fluid phase of the
%% form,

%% \begin{equation}
%% \rho^{\mathrm{f}} \bq^{\mathrm{f}} = - \bD^{\mathrm{f}}
%% \bv^{\mathrm{f}},
%% \label{eu-force-rel-fluid}
%% \end{equation}

%% \noindent where  $\bD^{\mathrm{f}}$ is a positive semi-definite
%% frictional coefficient tensor, the remaining portion of the
%% dissipation inequality for the fluid is:

%% Constitutively prescribing the fluid stress relation
%% (\ref{eu-stress-rel-fluid}) and an interaction force on the fluid as 
%% in (\ref{eu-force-rel-fluid}) results in a priori satisfaction of all
%% the fluid-derived terms reduced dissipation inequality
%% (\ref{reduced-dissipation-fluid}).

%% Decomposing the growth deformation gradient tensor into its
%% dilatational and volume-preserving parts, $\bF^{\mathrm{g}} =
%% J^{\mathrm{g}^{1/3}} \bar{\bF}^{\mathrm{g}}$, the above inequality
%% becomes,

%% \begin{equation*}
%% - \bF^{\mathrm{e}^{\mathrm{T}}} \bP^{\mathrm{c}} \colon
%% \dot{\bF^{\mathrm{g}}} \leq 0,
%% \end{equation*}

%% However, since growth does not take place at the time scales
%% mechanical tests are performed, calculations in Sections which show
%% the mechanical behaviour of the model under various cases assumes
%% fg = 1 and fgdot = 0, which also obey the above inequality. This term
%% will be reintroduced when studying the mechanics of growing tumours in
%% section.
