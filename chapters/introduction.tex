\chapter{Introduction}
\label{introduction}

This dissertation presents a continuum treatment of growth in
biological tissue developed within the context of modern mixture
theory. The crux of this work is a careful examination of the
assumptions underlying continuum thermodynamics under the condition
that multiple interacting species occupy a region of Euclidean space
simultaneously. The formal axiomatic treatment presented derives from
these assumptions, and provides insight into the sequence of
interaction between tissue mechanics, mass transport and biochemical
reactions. A computational formulation built upon the theory is used
to solve a broad class of numerical examples demonstrating several
biophysical aspects of tissue growth.

The first portion of this initial chapter (Section~\ref{background})
provides some context for this work and the second
(Section~\ref{overview}) provides an overview of the topics considered
in the remainder of the dissertation.

\section{Background and motivation}
\label{background}

\emph{Growth} involves the addition or depletion of mass in biological
tissue. Growth occurs in combination with \emph{remodelling}, which is
a change in microstructure, and possibly with \emph{morphogenesis},
which is a change in form in the embryonic state. The physics of these
processes are quite distinct, and for modelling purposes can, and
must, be separated. In this work, biological growth is formulated on a
continuum scale within the context of mixture theory
\citep{TruesdellToupin:60, TruesdellNoll:65, BedfordDrumheller:1983},
which allows us to systematically account for the numerous interacting
and inter-converting species constituting the tissue. The crux of this
work is a careful examination of the assumptions underlying continuum
thermodynamics for these mixtures, especially in the presence of
supplementary terms which enhance balance laws from classical
mechanics to allow for the complex behaviour of tissues.

There have been a number of significant papers on biological growth
(and remodelling) in the past decade, of which we touch upon some whose
approaches are either similar to this work in some respects, or differ
in important ways.

Mass transport is important to the growth problem, and
\citet{EpsteinMaugin:2000} introduced a mass flux term to the
corresponding transport equation. In their work, they also considered irreversible
momentum and entropy contributions from the species flux to account
for these aspects of the inter-species interactions.

Recognising the importance of mass transport to the
growth problem, introduced a mass flux term
to their transport equation. Their work also introduced , and deduces
non-symmetric partial Cauchy stresses, in contrast to the treatment here. (See
Section~\ref{balance-of-angular-momentum}).

\citet{HumphreyRajagopal:02} provided a
mathematical treatment of \emph{adaptation} in a tissue, which
includes growth and remodelling in the sense of this work. They
introduced the notion of evolving natural configurations to model the
state of material deposited at different instants in time. The
treatment of the growth part of the deformation gradient in this work
(Section~\ref{kinematics-of-growth}) bears some resemblance to this
idea. This concept also forms the basis for an active active field of
study within the literature \citep{Skalak:81,
  SkalakHoger:96, Klischetal:2001, TaberHumphrey:2001,
  LubardaHoger:02, AmbrosiMollica:2002} focusing on the kinematic
aspects of on biological growth.

\citet{PreziosiFarina:2002} developed an extension to the classical
Darcy's Law to incorporate mass exchanges between reacting
species. This consideration is relevant to growth problems; however,
these issues were subsumed in \citet{growthpaper}, upon which this
work is based.

Finally, we turn to some recent works on the area. HMB, BL, GAT, And
they all have the same balance laws, plus subsequently varying choices made in the
different works, including this one, for the constitutive independent
variables result in altered constitutive specification. PLUS, the
choices in this work are {\em actually} fleshed out and solved
retaining much of their complexity. It is not just a formulation.

 Many of the ideas 
employed here are applicable to tumour growth problems; however, due
to our current focus on tendon, we do not include phenomena
such as angiogenesis and cell migration \citep[see for
  example][]{Brewardetal:2003}.

Perhaps a bit surprisingly the above body of literature, except for
Ambrosi and Mollica (2002), has steered clear of the application to
cancer, despite the obvious centrality of growth and resorption to the
physical modelling of solid tumors. The preference has been for
healthy and pathological growth of bones, and soft tissue. The
literature on cancer modelling in mathematical biology has focused on
time evolution of populations, while ignoring spatial variations (for
example Kuznetsov et al., 1994; d'Onofrio, 2005). Recently, however,
there has been some work on tumor modelling that has included spatial
variations by accounting for mechanical effects, such as Jackson and
Byrne (2002) and the more recent review on cancer dynamics by Byrne et
al. (2006). These works include a multi-species formulation with
reactions, mass transport and mechanics, and are therefore more
aligned with Garikipati et al. (2004) and Sengers et al. (2004).

\section{An overview}
\label{overview}

The core of this dissertation is divided into two parts.

The first part, consisting of Chapters~\ref{lagrangian-perspective}
and \ref{numerical-simulations-1}, develops the theoretical
formulation for biological growth from a Lagrangian perspective and
presents representative numerical examples demonstrating aspects of
the coupled physics using a corresponding computational
implementation. This approach, based on our previous work
\citep{growthpaper}, draws in some measure from
\citet{CowinHegedus:76, EpsteinMaugin:2000} and
\citet{TaberHumphrey:2001}, and works in terms of material quantities
defined in the reference configuration of the tissue.

The theoretical treatment presented in Chapters~\ref{lagrangian%
  -perspective} begins by deriving general balance equations governing
the behaviour multi-phase mixtures, and then proceeds to specify
constitutive relationships pertinent to growing biological tissue that
are thermody\-namically-consistent, in the sense that specification of
these relations does not violate the Clausius-Duhem dissipation
inequality. Two important contributions of this work include a
comprehensive account of the coupling between transport and mechanics
(stemming from from the balance equations, kinematics and constitutive
relations), and an improvement to the mathematical treatment that
allows for the numerical stabilisation of the advection-diffusion mass
transport equation in the advection-dominated regime.

This approach was impaired by some basic deficiencies. Firstly, while
the transport equations were posed (consistently) reference
configuration, for a tissue undergoing finite strains, the physics of
fluid-tissue interactions and the imposition of relevant boundary
conditions is best understood and represented in the current
configuration.
Secondly, also stemming from its 
roots in solid mechanics, the formulation relied upon primitive
quantities that are not natural to fluids, such as the {\em
  deformation gradient of the fluid}. While such quantities can be
formally defined, they are not easily tracked during the course of
solving boundary value problems. One final problem with this approach
arose from attempting to impose the balance of momentum for the tissue
as a whole, and this required additional assumptions on the
microstructural mechanics. These require sophisticated homogenisation
techniques (e.g., \citet{IdiartCastaneda:2003}) and have strong implications
for the stiffness of tissue response, the nature of fluid transport,
and since nutrients are dissolved in the fluid, ultimately for
growth. This meant that without additional, complex assumptions, the
formulation could not provide accurate results. (The calculations in
Section~\ref{constriction-1}, however, do work determine the upper and
lower bounds of the solutions.)

It is this that motivates the second part of this dissertation,
composed of Chapters~\ref{eulerian-perspective} and
\ref{numerical-simulations-2}.


%% Mention the fact that the boundary conditions we work with in the
%% second formulation is much more natural and amenable to
%% experiments.

Chapter~\ref{eulerian-perspective} Refining our understanding over these few years, recognising that
there exist many fluid-like phases in the tissue, and they have to be
treated fundamentally different--in terms of their own primitive
variables (velocities, pressure), we revised and further enhanced the
formulation by infusing few years worth of insight gathered about the
system. This discussion is at the heart of Chapter 4.

 With the revised formulation exists the related computational
framework, now written to solve the detailed set of balance equations
and not rely on simplifying assumptions. This seems promising in that
it demonstrates several basic aspects of the physics of these problems
and has been relatively easy to extend to varied classes of problems,
as evidenced by the examples in Chapter 5. it works well for our
tissues of interest.

The computational framework presents a powerful tool to answer
specific questions—ranging from those pertinent to viscoelastic
aspects of the mechanical response of growing tendons under different
loading conditions, to quantitative investigations of the efficacy of
drugs based on how they are administered, to understanding the
cellular processes associated with tumour growth.

Chapter 6 concludes, As is their very nature, the appendices at the end of this thesis
house topics that while useful, and sometimes interesting, interrupt
the flow of ideas during the main
discourse.

% Local Variables:
% TeX-master: "thesis"
% mode: latex
% mode: flyspell
% End:
