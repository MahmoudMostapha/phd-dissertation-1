\chapter{Representative numerical simulations II}
\label{numerical-simulations-2}

\todo{\begin{itemize}
  \item Work on this after, and in tandem with, Chapter 4
  \item Weave around the monolithic implementation details and refer
    COMSOL's documentation to see how it goes about doing things?
    (Cite back to the fact that monolithic is stable.)
  \item Figure out the feasibility of: elastica, visco-poro-elasticity
    friction-coeff. variation or drop the entire section.
  \item Generate pristine plots for the flow field calculations and
    write up elaborate descriptions of the problems, citing back to
    the boundary conditions/saturation constraint of the previous
    chapter. Justify 2D.
  \item Bring in a few experimental figures and point out what we're
    looking for in terms of biphasic response.
  \item Go through cool/realistic calculations and replot a few from
    useful cases. Look at the new proposal to see what is sellable.
  \item Cite work from cancer to motivate the example. Generate many
    plots of the different fields and describe in great detail the toy
    problem citing back to Onsager reciprocity.
\end{itemize}}

\section{The coupled solution scheme}
\label{coupled-solution-scheme-2}

The second set of numerical simulations reside here.

\section{Material modelling}
\label{material-modelling}

\subsection{Elastica-based strain energy}
\label{elastica-stain-energy}

\subsection{Non-linear viscoelasticity}
\label{non-linear-viscoelasticity}

\subsection{Friction coefficient variation with concentration}
\label{variable-friction-coefficient}

\section{Simple physical tests}
\label{simple-physics}

\subsection{Balloon}
\label{balloon}

\begin{figure}
\centering
{\includegraphics[width=0.8\textwidth]{images/examples/eulerian/swelling/balloon-swell-0p0}}
\caption{The swelling balloon.} 
\label{swelling-balloon-image-0}
\end{figure}

\begin{figure}
\centering
{\includegraphics[width=0.8\textwidth]{images/examples/eulerian/swelling/balloon-swell-0p6}}
\caption{The swelling balloon.} 
\label{swelling-balloon-image-1}
\end{figure}

\begin{figure}
\centering
{\includegraphics[width=0.8\textwidth]{images/examples/eulerian/swelling/balloon-swell-1p2}}
\caption{The swelling balloon.} 
\label{swelling-balloon-image-2}
\end{figure}

\begin{figure}
\centering
{\includegraphics[width=0.8\textwidth]{images/examples/eulerian/swelling/balloon-swell-1p8}}
\caption{The swelling balloon.} 
\label{swelling-balloon-image-3}
\end{figure}

\begin{figure}
\centering
{\includegraphics[width=0.8\textwidth]{images/examples/eulerian/swelling/balloon-swell-2p4}}
\caption{The swelling balloon.} 
\label{swelling-balloon-image-4}
\end{figure}

\begin{figure}
\centering
{\includegraphics[width=0.8\textwidth]{images/examples/eulerian/swelling/balloon-swell-3p0}}
\caption{The swelling balloon.} 
\label{swelling-balloon-image-5}
\end{figure}

\subsection{Flow under tension}
\label{tenson-flow}

\subsection{Constriction}
\label{constriction-2}

\section{Examples exploring the biphasic nature of porous soft tissue}
\label{biphasic-examples-2}

\subsection{Stress relaxation}
\label{stress-relaxation}
With viscoelasticity and poroelasticity, quasistatic and dynamic.

\subsection{Hysteresis}
\label{hysteresis}
With viscoelasticity and poroelasticity, quasistatic and dynamic.

\section{Tumour growth}
\label{tumor-growth}

%

% Local Variables:
% TeX-master: "thesis"
% mode: latex
% mode: flyspell
% End:
